% Options for packages loaded elsewhere
\PassOptionsToPackage{unicode}{hyperref}
\PassOptionsToPackage{hyphens}{url}
\PassOptionsToPackage{dvipsnames,svgnames,x11names}{xcolor}
%
\documentclass[
]{scrbook}
\usepackage{amsmath,amssymb}
\usepackage{iftex}
\ifPDFTeX
  \usepackage[T1]{fontenc}
  \usepackage[utf8]{inputenc}
  \usepackage{textcomp} % provide euro and other symbols
\else % if luatex or xetex
  \usepackage{unicode-math} % this also loads fontspec
  \defaultfontfeatures{Scale=MatchLowercase}
  \defaultfontfeatures[\rmfamily]{Ligatures=TeX,Scale=1}
\fi
\usepackage{lmodern}
\ifPDFTeX\else
  % xetex/luatex font selection
\fi
% Use upquote if available, for straight quotes in verbatim environments
\IfFileExists{upquote.sty}{\usepackage{upquote}}{}
\IfFileExists{microtype.sty}{% use microtype if available
  \usepackage[]{microtype}
  \UseMicrotypeSet[protrusion]{basicmath} % disable protrusion for tt fonts
}{}
\makeatletter
\@ifundefined{KOMAClassName}{% if non-KOMA class
  \IfFileExists{parskip.sty}{%
    \usepackage{parskip}
  }{% else
    \setlength{\parindent}{0pt}
    \setlength{\parskip}{6pt plus 2pt minus 1pt}}
}{% if KOMA class
  \KOMAoptions{parskip=half}}
\makeatother
\usepackage{xcolor}
\usepackage{longtable,booktabs,array}
\usepackage{calc} % for calculating minipage widths
% Correct order of tables after \paragraph or \subparagraph
\usepackage{etoolbox}
\makeatletter
\patchcmd\longtable{\par}{\if@noskipsec\mbox{}\fi\par}{}{}
\makeatother
% Allow footnotes in longtable head/foot
\IfFileExists{footnotehyper.sty}{\usepackage{footnotehyper}}{\usepackage{footnote}}
\makesavenoteenv{longtable}
\usepackage{graphicx}
\makeatletter
\def\maxwidth{\ifdim\Gin@nat@width>\linewidth\linewidth\else\Gin@nat@width\fi}
\def\maxheight{\ifdim\Gin@nat@height>\textheight\textheight\else\Gin@nat@height\fi}
\makeatother
% Scale images if necessary, so that they will not overflow the page
% margins by default, and it is still possible to overwrite the defaults
% using explicit options in \includegraphics[width, height, ...]{}
\setkeys{Gin}{width=\maxwidth,height=\maxheight,keepaspectratio}
% Set default figure placement to htbp
\makeatletter
\def\fps@figure{htbp}
\makeatother
\setlength{\emergencystretch}{3em} % prevent overfull lines
\providecommand{\tightlist}{%
  \setlength{\itemsep}{0pt}\setlength{\parskip}{0pt}}
\setcounter{secnumdepth}{5}
\newlength{\cslhangindent}
\setlength{\cslhangindent}{1.5em}
\newlength{\csllabelwidth}
\setlength{\csllabelwidth}{3em}
\newlength{\cslentryspacingunit} % times entry-spacing
\setlength{\cslentryspacingunit}{\parskip}
\newenvironment{CSLReferences}[2] % #1 hanging-ident, #2 entry spacing
 {% don't indent paragraphs
  \setlength{\parindent}{0pt}
  % turn on hanging indent if param 1 is 1
  \ifodd #1
  \let\oldpar\par
  \def\par{\hangindent=\cslhangindent\oldpar}
  \fi
  % set entry spacing
  \setlength{\parskip}{#2\cslentryspacingunit}
 }%
 {}
\usepackage{calc}
\newcommand{\CSLBlock}[1]{#1\hfill\break}
\newcommand{\CSLLeftMargin}[1]{\parbox[t]{\csllabelwidth}{#1}}
\newcommand{\CSLRightInline}[1]{\parbox[t]{\linewidth - \csllabelwidth}{#1}\break}
\newcommand{\CSLIndent}[1]{\hspace{\cslhangindent}#1}
\ifLuaTeX
\usepackage[bidi=basic]{babel}
\else
\usepackage[bidi=default]{babel}
\fi
\babelprovide[main,import]{ngerman}
% get rid of language-specific shorthands (see #6817):
\let\LanguageShortHands\languageshorthands
\def\languageshorthands#1{}
\addtokomafont{disposition}{\rmfamily}
\KOMAoptions{numbers=noenddot}

\usepackage{libertine}
\usepackage{libertinus-otf}

\usepackage[original]{imakeidx}
% \usepackage{makeidx}
% \makeindex[title=Stichwortverzeichnis, intoc]
% \indexsetup{level=\chapter,toclevel=chapter,noclearpage,firstpagestyle=headings}



\usepackage{color}
\definecolor{formalColor}{HTML}{00A2FF}
\definecolor{semanticColor}{HTML}{1DB100}
\definecolor{concreteColor}{HTML}{EE220C}
\definecolor{empiricColor}{HTML}{F8BA00}
\definecolor{linkColor}{HTML}{929292}
\definecolor{quoteColor}{HTML}{666666}

\usepackage{framed}
\renewenvironment{quote}{
  \list{}{
	\leftmargin0.2cm   % this is the adjusting screw
    \rightmargin\leftmargin
      	\def\FrameCommand
    {%
        {\color{quoteColor}\vrule width 2pt}%
        \hspace{0pt}%must no space.
        %
    }%
    \MakeFramed{\advance \hsize -\width \FrameRestore}    \color{quoteColor}
    }
  \item\relax
}
{\endlist\color{black}\endMakeFramed}


\makeatletter
\def\renewtheorem#1{%
  \expandafter\let\csname#1\endcsname\relax
  \expandafter\let\csname c@#1\endcsname\relax
  \gdef\renewtheorem@envname{#1}
  \renewtheorem@secpar
}
\def\renewtheorem@secpar{\@ifnextchar[{\renewtheorem@numberedlike}{\renewtheorem@nonumberedlike}}
\def\renewtheorem@numberedlike[#1]#2{\newtheorem{\renewtheorem@envname}[#1]{#2}}
\def\renewtheorem@nonumberedlike#1{
\def\renewtheorem@caption{#1}
\edef\renewtheorem@nowithin{\noexpand\newtheorem{\renewtheorem@envname}{\renewtheorem@caption}}
\renewtheorem@thirdpar
}
\def\renewtheorem@thirdpar{\@ifnextchar[{\renewtheorem@within}{\renewtheorem@nowithin}}
\def\renewtheorem@within[#1]{\renewtheorem@nowithin[#1]}
\makeatother
\ifLuaTeX
  \usepackage{selnolig}  % disable illegal ligatures
\fi
\IfFileExists{bookmark.sty}{\usepackage{bookmark}}{\usepackage{hyperref}}
\IfFileExists{xurl.sty}{\usepackage{xurl}}{} % add URL line breaks if available
\urlstyle{same}
\hypersetup{
  pdftitle={Stoffdidaktik Mathematik -- Skript zur Vorlesung im Wintersemester 2023/24},
  pdfauthor={Dr.~Heiko Etzold, Universität Potsdam},
  pdflang={de-DE},
  colorlinks=true,
  linkcolor={linkColor},
  filecolor={Maroon},
  citecolor={Blue},
  urlcolor={linkColor},
  pdfcreator={LaTeX via pandoc}}

\title{Stoffdidaktik Mathematik -- Skript zur Vorlesung im Wintersemester 2023/24}
\author{Dr.~Heiko Etzold, Universität Potsdam}
\date{Letzte Änderung: 08.12.2023}

\usepackage{amsthm}
\newtheorem{theorem}{Theorem}[chapter]
\newtheorem{lemma}{Lemma}[chapter]
\newtheorem{corollary}{Corollary}[chapter]
\newtheorem{proposition}{Proposition}[chapter]
\newtheorem{conjecture}{Conjecture}[chapter]
\theoremstyle{definition}
\newtheorem{definition}{Definition}[chapter]
\theoremstyle{definition}
\newtheorem{example}{Example}[chapter]
\theoremstyle{definition}
\newtheorem{exercise}{Exercise}[chapter]
\theoremstyle{definition}
\newtheorem{hypothesis}{Hypothesis}[chapter]
\theoremstyle{remark}
\newtheorem*{remark}{Remark}
\newtheorem*{solution}{Solution}
\begin{document}
\maketitle

% \renewtheorem{definition}{Definition}[chapter]
%
% \newtheoremstyle{definition}% name of the style to be used
% {}% measure of space to leave above the theorem. E.g.: 3pt
% {}% measure of space to leave below the theorem. E.g.: 3pt
% {\em}% name of font to use in the body of the theorem
% {}% measure of space to indent
% {\bf}% name of head font
% {.}% punctuation between head and body
% { }% space after theorem head; " " = normal interword space
% {\thmname{#1}\thmnumber{\addtocounter{thm}{-1} #2$^\prime$}\thmnote{\textnormal{ (#3)}}}

{
\hypersetup{linkcolor=}
\setcounter{tocdepth}{1}
\tableofcontents
}
\hypertarget{uxfcber-dieses-dokument}{%
\chapter*{Über dieses Dokument}\label{uxfcber-dieses-dokument}}
\addcontentsline{toc}{chapter}{Über dieses Dokument}

Die Veranstaltung \emph{Stoffdidaktik Mathematik} wird über dieses Dokument begleitet. Es wird fortlaufend aktualisiert und zur Verfügung gestellt. Über ein Git-Respository können Änderungen nachverfolgt werden. In der html-Version gelangt man über die Menüleiste am oberen Rand sowohl zu den Rohdaten als auch zu einer pdf-Version. Die Darstellung der Inhalte ist jedoch optimiert für die html-Version dieses Dokuments.

Aufgrund der permanenten Entwicklung ist eine Zitation des aktuellen Skriptes nicht unbedingt geeignet. Sollte ein Verweis notwendig sein, wird als Quellenangabe empfohlen:

\begin{quote}
Etzold, H. (2023). \emph{Stoffdidaktik Mathematik -- Skript zur Vorlesung im Wintersemester 2023/24} (Version vom 08.12.2023). \url{https://stoffdidaktik.heiko-etzold.de}
\end{quote}

Die Vorlesungsskripte der letztjährigen Veranstaltungen, die sich dann auch zur Zitation eignen, finden Sie unter:

\begin{itemize}
\tightlist
\item
  \url{https://stoffdidaktik.heiko-etzold.de/2022}
\item
  \url{https://stoffdidaktik.heiko-etzold.de/2021}
\end{itemize}

\hypertarget{lizenz}{%
\section*{Lizenz}\label{lizenz}}

Soweit nicht anders gekennzeichnet, ist dieses Dokument unter einem Creative Commons Lizenzvertrag lizenziert: »Namensnennung -- Weitergabe unter gleichen Bedingungen 4.0 International«. Dies gilt nicht für Zitate und Werke, die aufgrund einer anderen Erlaubnis genutzt werden.
Eine Beschreibung der Lizenz findet sich unter \url{https://creativecommons.org/licenses/by-sa/4.0/deed.de}.

Ausgenommen von der CC-BY-SA-Lizenz sind insbesondere die Abbildungen \ref{fig:FreudenthalWinkel} und \ref{fig:FreudenthalWinkelSpiegeln} -- diese werden im Sinne des Zitaterechts (\href{https://www.gesetze-im-internet.de/urhg/__51.html}{§~51 UrhG}) verwendet.

\hypertarget{stoffdidaktik-mathematik-an-der-up}{%
\chapter*{Stoffdidaktik Mathematik an der UP}\label{stoffdidaktik-mathematik-an-der-up}}
\addcontentsline{toc}{chapter}{Stoffdidaktik Mathematik an der UP}

\hypertarget{struktur-der-veranstaltung}{%
\section*{Struktur der Veranstaltung}\label{struktur-der-veranstaltung}}
\addcontentsline{toc}{section}{Struktur der Veranstaltung}

Die Veranstaltung \emph{Stoffdidaktik Mathematik}\footnote{Die Modulbeschreibung finden Sie bei \href{https://puls.uni-potsdam.de/qisserver/rds?state=verpublish\&status=init\&vmfile=no\&moduleCall=modulansicht\&publishConfFile=modulverwaltung\&publishSubDir=up/modulbearbeiter\&\&modul.modul_id=3155\&menuid=\&topitem=Modulbeschreibung\&subitem=}{PULS}.} besteht aus einer \textbf{Vorlesung (2~SWS)} und einem zugehörigen \textbf{Seminar (2~SWS)}.

Im Wintersemester 2023/24 wird die \textbf{Vorlesung semesterbegleitend} kompakt in der zweiten Semesterhälfte stattfinden. Das \textbf{Seminar} können Sie entweder \textbf{parallel zur Vorlesung} im Wintersemester oder \textbf{semesterbegleitend} im Sommersemester 2024 besuchen.

In der Vorlesung erhalten Sie einen \textbf{Input zu stoffdidaktischen Grundlagen}, wobei der Schwerpunkt auf \textbf{stoffdidaktischen Theorien} liegt, die über vielfältige Unterrichtsbeispiele illustriert werden. Im Seminar haben Sie die Aufgabe, diese Grundlagen selbstständig \textbf{auf verschiedene Lerngegenstände anzuwenden}.

Sie halten einen \textbf{Seminarvortrag} (30 bis 45 Minuten) als Voraussetzung für die Zulassung zur Modulprüfung und fassen Ihre Erarbeitungen in einer \textbf{Hausarbeit} (6 bis 8 Seiten) zusammen, die als Modulprüfung dient.

\hypertarget{einordnung}{%
\section*{Einordnung}\label{einordnung}}
\addcontentsline{toc}{section}{Einordnung}

Die Veranstaltung \emph{Stoffdidaktik Mathematik} findet nach empfohlenem Studienverlaufsplan im \textbf{5. Fachsemester parallel zur \emph{Einführung in die Mathematikdidaktik}} statt.

Das heißt insbesondere, dass Sie bereits die \textbf{Grundlagen} zur Analysis, Linearen Algebra, Stochastik, Geometrie, Algebra und Numerik studiert haben sollten. Auf diese Grundlagen wird in der Stoffdidaktisch \textbf{fachlich aufgebaut}.

Während Sie sich in der \emph{Einführung in die Mathematikdidaktik} mit verschiedenen Lehr-Lern-Theorien, Unterrichtsprinzipien, prozessbezogenen Kompetenzen oder methodischen Grundlagen des Mathematikunterrichtens beschäftigen, liegt in der \emph{Stoffdidaktik Mathematik} der Fokus auf der \textbf{Auswahl und Strukturierung der Unterrichtsinhalte}, basierend auf fachlichen und fachdidaktischen Erkenntnissen. Im Wintersemester 2023/24 wird durch eine kompakte Durchführung von Einführung (1. Semesterhälfte) und Stoffdidaktik (2. Semesterhälfte) die Einführungsvorlesung zeitlich vor der Stoffdidaktikvorlesung stattfinden.

Parallel oder im Anschluss an beide Veranstaltungen absolvieren Sie das \textbf{fachdidaktische Tagespraktikum}, in dem Sie die erworbenen Kenntnisse in die \textbf{Unterrichtspraxis} transferieren und erste eigene Unterrichtsstunden im Fach Mathematik halten.

\hypertarget{kompetenzziele-der-veranstaltung}{%
\section*{Kompetenzziele der Veranstaltung}\label{kompetenzziele-der-veranstaltung}}
\addcontentsline{toc}{section}{Kompetenzziele der Veranstaltung}

Als Kompetenzen, die Sie nach Abschluss von Vorlesung und Seminar erreicht haben sollen, sind angedacht:

\begin{itemize}
\tightlist
\item
  Sie \textbf{kennen Aspekte und Grundvorstellungen} zu zentralen mathematischen Begriffen.
\item
  Sie \textbf{beurteilen Unterrichtsmaterialien und Lernumgebungen} hinsichtlich ihrer stoffdidaktischen Eignung.
\item
  Sie \textbf{erstellen Aufgaben und erste Lernumgebungen} zu konkreten Stoffgebieten.
\item
  Sie \textbf{erkennen mathematikdidaktische Prinzipien und Ideen} als \textbf{Entscheidungs- und Strukturierungsgrundlage} zu stofflichen Inhalten der mathematischen Bildung.
\item
  Sie \textbf{wählen zielgerichtet} analoge und digitale \textbf{Medien} zur Unterstützung stofflich orientierter Lehr-Lern-Prozesse aus.
\item
  Sie \textbf{setzen sich} selbstständig \textbf{mit stoffdidaktischen Fragestellungen auseinander} und nutzen dafür geeignete mathematikdidaktische Literatur.
\item
  Sie \textbf{reflektieren die Inhalte der vorangegangenen Mathematik-Fachmodule} unter stoffdidaktischen Gesichtspunkten.
\end{itemize}

\hypertarget{was-ist-stoffdidaktik}{%
\section*{Was ist Stoffdidaktik?}\label{was-ist-stoffdidaktik}}
\addcontentsline{toc}{section}{Was ist Stoffdidaktik?}

Für die Disziplin der \emph{Stoffdidaktik Mathematik} gibt es keine allgemeingültige Definition, auch haben sich die Schwerpunkte in der historischen Entwicklung stets verschoben.

Grundsätzliches Ziel ist, stoffliche Inhaltsbereiche für den Mathematikunterricht auszuwählen (\textbf{\emph{Was?}}) und aufzubereiten (\textbf{\emph{Wie?}}). Im Sinne dieser Veranstaltung kann Stoffdidaktik grob als \textbf{Spezifierung und Strukturierung von Lerngegenständen} aufgefasst werden (zur begrifflichen Einordnung siehe auch \protect\hyperlink{ref-Hussmann:2016a}{Hußmann et al., 2016}).

Während hierzu, historisch betrachtet, anfangs der Stoff ausschließlich aus fachmathematischer Perspektive aufbereitet wurde (z.~B. durch \emph{didaktisch-orientierte Sachanalysen}), nahmen in der Folgezeit mehr und mehr auch Lehr-Lern-Theorien Einzug -- gar ein Verschwinden der stofflichen Orientierung der Mathematikdidaktik wird befürchtet (vgl. \protect\hyperlink{ref-Jahnke:2010}{Jahnke, 2010}).

Mit dem Begriff der \textbf{Strukturgenetischen Analyse} erweitert Wittmann (\protect\hyperlink{ref-Wittmann:2015}{2015}) die historische Sichtweise als eine »Mathematikdidaktik \emph{vom Fach aus}«, die sich »auf implizite Theorien des Lehrens und Lernens, die im Fach selbst liegen{[}, stützt{]}« (\protect\hyperlink{ref-Wittmann:2015}{Wittmann, 2015, S. 240}). »Anders als bei der Stoffdidaktik, die sich im Wesentlichen auf die logische Analyse des Stoffes und die Wissensvermittlung konzentriert hat, stehen jetzt aber die Genese des Wissens im Verlauf der Schulzeit und Lernprozesse unter Bezug auf unterschiedliche Lernvoraussetzungen im Vordergrund« (\protect\hyperlink{ref-Wittmann:2015}{Wittmann, 2015, S. 250}). Eine derartig ganzheitliche Sichtweise soll auch den Geist dieser Veranstaltung ausmachen.

\begin{quote}
\textbf{Überblicke zur historischen Entwicklung der Stoffdidaktik}

\begin{itemize}
\tightlist
\item
  Hefendehl-Hebeker (\protect\hyperlink{ref-Hefendehl-Hebeker:2016}{2016}): Subject-matter didactics in German traditions: Early historical developments
\item
  Schupp (\protect\hyperlink{ref-Schupp:2016}{2016, 79~ff.}): Gedanken zum „Stoff`` und zur „Stoffdidaktik`` sowie zu ihrer Bedeutung für die Qualität des Mathematikunterrichts
\end{itemize}
\end{quote}

\hypertarget{part-stoffdidaktische-analyse}{%
\part*{Stoffdidaktische Analyse}\label{part-stoffdidaktische-analyse}}
\addcontentsline{toc}{part}{Stoffdidaktische Analyse}

\hypertarget{vier-ebenen-ansatz}{%
\chapter{Vier-Ebenen-Ansatz}\label{vier-ebenen-ansatz}}

\begin{quote}
\textbf{Ziele}

\begin{itemize}
\tightlist
\item
  Sie kennen typische Fragestellungen, um sich einer stoffdidaktischen Analyse systematisch zu nähern.
\item
  Sie erkennen den Vier-Ebenen-Ansatz als eine Möglichkeit, eine stoffdidaktische Analyse strukturiert vorzunehmen.
\item
  Sie können den Vier-Ebenen-Ansatz anhand eines Beispiels nachvollziehen.
\item
  Sie sind sich der Komplexität einer stoffdidaktischen Analyse bewusst.
\end{itemize}

\textbf{Material}

\begin{itemize}
\tightlist
\item
  Folien zum Kapitel 1 (\href{files/Stoffdidaktik2023-01-VierEbenenAnsatz.pdf}{pdf}, \href{files/Stoffdidaktik2023-01-VierEbenenAnsatz.key}{Keynote})
\item
  \href{https://apps.apple.com/de/app/winkel-farm/id1369585218}{App \emph{Winkel-Farm}} (nur für iOS)
\end{itemize}
\end{quote}

\hypertarget{analyse-von-lerngegenstuxe4nden}{%
\section{Analyse von Lerngegenständen}\label{analyse-von-lerngegenstuxe4nden}}

Die inhaltliche Basis der Veranstaltung bietet ein Beitrag von Hußmann \& Prediger (\protect\hyperlink{ref-Hussmann:2016}{2016}) zur Spezifizierung und Strukturierung mathematischer Lerngegenstände\index{Lerngegenstand}. Nur einen Artikel als Basis einer kompletten 4~SWS starken Veranstaltung zu nutzen, scheint zunächst unüblich. In diesem Fall ist es jedoch hilfreich, da der Beitrag eine Kategorisierung stoffdidaktischer Analysen vorschlägt und vielfältige Fragen formuliert, woraus sich wieder ein ganzes Repertoir an Themen ergibt, die es im Rahmen von Vorlesung und Seminar zu untersuchen gilt.

Hußmann \& Prediger (\protect\hyperlink{ref-Hussmann:2016}{2016, 35~f.})\index{4-Ebenen-Ansatz|see{Vier-Ebenen-Ansatz}} kategorisieren eine stoffdidaktische Analyse in eine \textbf{\textcolor{formalColor}{formale}}, \textbf{\textcolor{semanticColor}{semantische}}, \textbf{\textcolor{concreteColor}{konkrete}} und \textbf{\textcolor{empiricColor}{empirische}} Ebene, wobei diese nicht hierarchisch aufgebaut sind, sondern sich gegenseitig beeinflussen. Innerhalb der Ebenen wird jeweils noch einmal in die \textbf{Spezifizierung} und die \textbf{Strukturierung} eines Lerngegenstands unterschieden.

Auf der \textcolor{formalColor}{formalen Ebene}\index{Vier-Ebenen-Ansatz!formale Ebene|textbf} wird der Lerngegenstand aus seiner fachlich-logischen Struktur heraus betrachtet.

Die \textcolor{semanticColor}{semantische Ebene}\index{Vier-Ebenen-Ansatz!semantische Ebene|textbf} adressiert Sinn und Bedeutung des mathematischen Gegenstands sowie erkenntnistheoretische Aspekte.

Ziel der \textcolor{concreteColor}{konkreten Ebene}\index{Vier-Ebenen-Ansatz!konkrete Ebene|textbf} ist die Umsetzung des Lehr-Lern-Prozesses an konkreten Situationen, über die das mathematische Wissen konstruiert wird.

Über die \textcolor{empiricColor}{empirische Ebene}\index{Vier-Ebenen-Ansatz!empirische Ebene|textbf} werden die kognitiven und ggf. sozialen Aspekte der Schülerinnen und Schüler in die stoffdidaktische Analyse integriert.

Über die \textbf{Spezifizierung}\index{Lerngegenstand!Spezifizierung|textbf} wird ermittelt, was genau Schülerinnen und Schüler bezüglich eines bestimmten mathematischen Themas lernen sollen, während die \textbf{Strukturierung}\index{Lerngegenstand!Strukturierung|textbf} analysiert, wie diese Elemente miteinander in Verbindung stehen und wie sie im Lernpfad strukturiert werden können.

Aus den vier Ebenen und der jeweiligen Unterscheidung in Spezifizierung und Strukturierung ergeben sich acht (nicht immer trennscharfe) Dimensionen, die den Analyseprozess zu einem Lerngegenstand kategorisieren können. Um dies für Forschungs- und Entwicklungsprozesse greifbar zu machen, haben Hußmann \& Prediger (\protect\hyperlink{ref-Hussmann:2016}{2016, S. 36}) typische Fragestellungen formuliert, an die in Tabelle \ref{tab:fragen-ebenen} angelehnt wird.

\begin{longtable}[]{@{}
  >{\raggedright\arraybackslash}p{(\columnwidth - 4\tabcolsep) * \real{0.2857}}
  >{\raggedright\arraybackslash}p{(\columnwidth - 4\tabcolsep) * \real{0.3571}}
  >{\raggedright\arraybackslash}p{(\columnwidth - 4\tabcolsep) * \real{0.3571}}@{}}
\caption{\label{tab:fragen-ebenen} Typische Fragestellungen, angelehnt an Hußmann \& Prediger (\protect\hyperlink{ref-Hussmann:2016}{2016, S. 36})}\tabularnewline
\toprule\noalign{}
\begin{minipage}[b]{\linewidth}\raggedright
\end{minipage} & \begin{minipage}[b]{\linewidth}\raggedright
Spezifizierung
\end{minipage} & \begin{minipage}[b]{\linewidth}\raggedright
Strukturierung
\end{minipage} \\
\midrule\noalign{}
\endfirsthead
\toprule\noalign{}
\begin{minipage}[b]{\linewidth}\raggedright
\end{minipage} & \begin{minipage}[b]{\linewidth}\raggedright
Spezifizierung
\end{minipage} & \begin{minipage}[b]{\linewidth}\raggedright
Strukturierung
\end{minipage} \\
\midrule\noalign{}
\endhead
\bottomrule\noalign{}
\endlastfoot
\textbf{\textcolor{formalColor}{Formale Ebene}} & \textbf{Welche Begriffe und Sätze} sollen erarbeitet werden? \textbf{Welche Verfahren} sollen erarbeitet werden und \textbf{wie} werden sie \textbf{formal begründet}? & Wie lassen sich die Begriffe, Sätze, Begründungen und Verfahren \textbf{logisch strukturieren}? Welche \textbf{Verbindungen} zwischen den Fachinhalten sind entscheidend, welche weniger bedeutsam? Wie kann das \textbf{Netzwerk} aus Begriffen, Sätzen, Begründungen und Verfahren entwickelt werden? \\
\textbf{\textcolor{semanticColor}{Semantische Ebene}} & \textbf{Welche Fundamentalen Ideen} liegen hinter den Begriffen, Sätzen und Verfahren? \textbf{Welche Grundvorstellungen und Repräsentationen} (graphisch, verbal, numerisch und algebraisch) sind für den Verständnisaufbau entscheidend? & Wie \textbf{verhalten} sich Ideen und Vorstellungen \textbf{zueinander} und \textbf{zu früheren und späteren Lerninhalten}? Wie kann ein \textbf{Lernpfad angeordnet} werden, in dem das Verständnis, zusammen mit den Erkenntnissen der formalen Ebene, aufgebaut wird? \\
\textbf{\textcolor{concreteColor}{Konkrete Ebene}} & \textbf{Welche Kernfragen und Kernideen} können die Entwicklung der Begriffe, Sätze und Verfahren leiten? \textbf{Welche Kontexte und Probleme} sind geeignet, um an ihnen die Kernfragen und -ideen exemplarisch zu behandeln und die Inhalte zu rekonstruieren? & Wie kann das Verständnis sukzessive \textbf{über konkrete Situationen} in den beabsichtigten Lernpfaden \textbf{konstruiert} werden (\emph{horizontale Mathematisierung})? Wie können die Lernpfade \textbf{in Bezug auf die Problemstruktur angeordnet} werden (\emph{vertikale Mathematisierung})? \\
\textbf{\textcolor{empiricColor}{Empirische Ebene}} & \textbf{Welche} typischen \textbf{individuellen Voraussetzungen} (Vorstellungen, Kenntnisse, Kompetenzen, \ldots) sind zu erwarten und \textbf{wie passen} diese zum \textbf{angestrebten Verständnis} (Ressourcen vs.~Hindernisse)? \textbf{Woher} kommen typische \textbf{Hindernisse} oder \textbf{unerwünschte Vorstellungen}? & Wie können typische \textbf{Vorkenntnisse und Vorstellungen} als \textbf{fruchtbare Anknüpfungspunkte} dienen? Welche \textbf{Schlüsselstellen} (Hindernisse, Wendepunkte, \ldots) gibt es \textbf{im Lernweg} der Schülerinnen und Schüler? Wie kann der angestrebte \textbf{Lernpfad} bezüglich der Anknüpfungspunkte und Schlüsselstellen \textbf{neu angeordnet} werden? \\
\end{longtable}

Diese Fragen können dabei helfen, einen Lerngegenstand aus professioneller Sicht vollumfänglich zu analysieren und daraus die Gestaltung eines Lernpfades für Schülerinnen und Schüler abzuleiten. Noch \emph{nicht} abgeleitet werden kann daraus jedoch die Gestaltung einer \emph{konkreten Unterrichtsstunde} -- dies bedarf weiterer Überlegungen, z.~B. zu Unterrichtsmethoden, Aufgaben, Klassenmanagement, \ldots{} (\protect\hyperlink{ref-Hussmann:2016}{Hußmann \& Prediger, 2016, S. 37}).

Als \textbf{Lerngegenstände} werden für spezifische Ausbildungszwecke ausgewählte Ausschnitte des gesellschaftlichen Wissens und Könnens angesehen (vgl. \protect\hyperlink{ref-Lompscher1985b}{Lompscher, 1985}). Diese Sichtweise entstammt tätigkeitstheoretischen Überlegungen, wobei zwischen gesellschaftlichem Wissen und Können und individuellen Kenntnissen, Fähigkeiten und Fertigkeiten unterschieden wird. Weitere Hintergrundinformationen dazu finden Sie in Abschnitt \ref{aneignung-von-lerngegenstaenden} sowie in den Kapiteln \ref{taetigkeitstheorie-und-lernen} und \ref{lernhandlungen-ausbilden}.

\hypertarget{beispiel-winkelbegriff}{%
\section{Beispiel Winkelbegriff}\label{beispiel-winkelbegriff}}

Um sich der Komplexität des Vier-Ebenen-Ansatzes bewusst zu werden, sollen mögliche Gedankengänge am Beispiel des Winkelbegriffs\index{Winkel|(} durchgeführt werden. Grundlage hierfür bietet die Dissertation \emph{Neue Zugänge zum Winkelbegriff} (\protect\hyperlink{ref-Etzold2021}{Etzold, 2021}). In dieser wird zwar nicht der Vier-Ebenen-Ansatz für die stoffdidaktische Analyse verfolgt, aber dennoch lassen sich die einzelnen Elemente darin wiederfinden. Ziel ist hier keine vollumfängliche stoffdidaktische Analyse, sondern eher eine Darstellung der exemplarischen Herangehensweise, wie man sich einer Spezifizierung und Strukturierung des Lerngegenstands \emph{Winkel} auf den vier Ebenen nähern kann.

\hypertarget{formale-ebene}{%
\subsection{Formale Ebene}\label{formale-ebene}}

Eine\index{Vier-Ebenen-Ansatz!formale Ebene|(} fachmathematische Analyse (bereits mit dem Blick auf eine schulische Nutzung) des Winkelbegriffs bieten u.~a. Freudenthal (\protect\hyperlink{ref-Freudenthal:1973}{1973b}), Strehl (\protect\hyperlink{ref-Strehl:1983}{1983}) oder Mitchelmore (\protect\hyperlink{ref-Mitchelmore:1990}{1990}).

Freudenthal (\protect\hyperlink{ref-Freudenthal:1973}{1973b, S. 441}) unterscheidet einen Winkel bspw. dahingehend, ob er über Geraden oder Halbgeraden (bzw. Strahlen) beschrieben wird, ob diese geordnet oder ungeordnet sind und ob sie in der orientierten oder unorientierten Ebene vorliegen (siehe Abbildung \ref{fig:FreudenthalWinkel}).



\begin{figure}

{\centering \includegraphics[width=0.75\linewidth]{pictures/1-FreudenthalWinkel} 

}

\caption{Winkelbegriffe nach Freudenthal (\protect\hyperlink{ref-Freudenthal:1973}{1973b, S. 441})}\label{fig:FreudenthalWinkel}
\end{figure}

Er diskutiert, welchen Einfluss die jeweilige Sichtweise auf dem Maßbereich hat, wie Winkel überhaupt gemessen werden können und wie mit Winkeln operiert werden kann. Was passiert denn, wenn man ein geordnetes Strahlenpaar in der orientierten Ebene spiegelt (vgl. \protect\hyperlink{ref-Freudenthal:1973}{Freudenthal, 1973b, 443~ff.})?

Wenn die Reihenfolge der Strahlen erhalten bleibt und die Winkelmessung aufgrund der Orientierung der Ebene vorgegeben ist, ändert sich damit ggf. auch das Maß des Winkels (siehe Abbildung \ref{fig:FreudenthalWinkelSpiegeln}).



\begin{figure}

{\centering \includegraphics[width=0.5\linewidth]{pictures/1-FreudenthalWinkelSpiegeln} 

}

\caption{Spiegelung eines goniometrischen Winkels (\protect\hyperlink{ref-Freudenthal:1973}{Freudenthal, 1973b, 443})}\label{fig:FreudenthalWinkelSpiegeln}
\end{figure}

Hierzu stellt Freudenthal (\protect\hyperlink{ref-Freudenthal:1973}{1973b, 443~ff.}) weitere fachmathematische Ausführungen dar und schließt damit, dass der elementargeometrische, goniometrische und analytische Winkelbegriff aus fachlicher Sicht für den schulischen Lernpfad unentbehrlich sind (\protect\hyperlink{ref-Freudenthal:1973}{Freudenthal, 1973b, S. 449}).

Die \emph{Spezifizierung} besteht also darin, den Begriff zu schärfen und Operationen mit ihm zu beschreiben. Die \emph{Strukturierung} besteht u.~a. in der vernetzenden Analyse der verschiedenen Winkelbegriffe und der Schlussfolgerung ihrer gleichermaßen Bedeutsamkeit für den Schulunterricht.\index{Vier-Ebenen-Ansatz!formale Ebene|)}

\hypertarget{semantische-ebene}{%
\subsection{Semantische Ebene}\label{semantische-ebene}}

Dazu,\index{Vier-Ebenen-Ansatz!semantische Ebene|(} welche Vorstellungen Schülerinnen und Schüler zum Winkelbegriff entwickeln sollen, sei u.~a. auf Krainer (\protect\hyperlink{ref-Krainer:1989}{1989}) und Mitchelmore \& White (\protect\hyperlink{ref-Mitchelmore:1998}{1998}) verwiesen. Eine grundsätzliche Schwierigkeit beim Unterrichten von Winkeln sind diverse und (scheinbar) nicht in Verbindung zu bringende Anwendungskontexte, die dennoch über denselben mathematischen Begriff beschrieben werden können. So ist das Sichtfeld eines Tieres ebenso wie die Umdrehung eines Wasserzählers über Winkel beschreibbar -- haben doch beide Situationen zunächst nichts miteinander zu tun.

Aufbauend auf den Arbeiten von Krainer (\protect\hyperlink{ref-Krainer:1989}{1989}) und Mitchelmore \& White (\protect\hyperlink{ref-Mitchelmore:1998}{1998}) können über eine Verknüpfung zur formalen Ebene mithilfe einer \emph{informationstheoretischen Winkeldefinition} (\protect\hyperlink{ref-Etzold2021}{Etzold, 2021, 39~f..}) vier Grundvorstellungen zum Winkelbegriff ausgearbeitet bzw. validiert werden:

\begin{itemize}
\tightlist
\item
  Winkel als Knick
\item
  Winkel als Feld
\item
  Winkel als Richtungsänderung
\item
  Winkel als Umdrehung
\end{itemize}

Dabei erhalten die \emph{Bestandteile} eines Winkels (Scheitelpunkt, Schenkel, ggf. Bereich zwischen den Schenkeln, Abweichungsmaß) eine besondere Bedeutung, über die sich auch eine sinnvolle Reihenfolge der Behandlung dieser Grundvorstellungen ableiten lässt. So »bietet es sich an, mit den Winkelfeldern zu beginnen. Bei diesen werden die meisten Bestandteile sichtbar (Scheitelpunkt, beide Schenkel als Begrenzungen sowie der zwischen den Schenkeln relevante Bereich) {[}\ldots{]}. Anschließend können Knicke oder Richtungsänderungen behandelt werden, woraufhin die Umdrehungen folgen.« (\protect\hyperlink{ref-Etzold2021}{Etzold, 2021, S. 60})

Die \emph{Spezifizierung} in diesem semantischen Teil ist demnach die Ausarbeitung der Grundvorstellungen. Die Begründung einer möglichen Reihenfolge kann der \emph{Strukturierung} des Lerngegenstands zugeordnet werden.\index{Vier-Ebenen-Ansatz!semantische Ebene|)}

\hypertarget{konkrete-ebene}{%
\subsection{Konkrete Ebene}\label{konkrete-ebene}}

Um\index{Vier-Ebenen-Ansatz!konkrete Ebene|(} die einzelnen Vorstellungen zu Winkeln aufzubauen, bedarf es charakteristischer Situationen, an denen der mathematische Kern der jeweiligen Vorstellung besonders gut sichtbar wird. Abbildung \ref{fig:Winkelsituationen} zeigt derartige \emph{Winkelsituationen} und die zugehörigen Grundvorstellungen (hier \emph{Winkelkontexte}).



\begin{figure}

{\centering \includegraphics[width=0.75\linewidth]{pictures/1-Winkelsituationen} 

}

\caption{Winkelsituationen und -kontexte (\protect\hyperlink{ref-Etzold2021}{Etzold, 2021, S. 70})}\label{fig:Winkelsituationen}
\end{figure}

Exemplarisch für die Grundvorstellung des Winkels als Feld wird darauf aufbauend eine Lernumgebung und darin eingebettetes Unterrichtsmaterial entwickelt, mithilfe dessen die Grundvorstellung ausgebildet werden kann. An der konkreten Situation der \emph{Sichtfelder von Tieren} sollen die Schülerinnen und Schüler Handlungen ausführen, die es ihnen ermöglicht, den mathematischen Kern hinter dem konkreten Beispiel zu erkunden.

Die Schülerinnen und Schüler nutzen dazu eine App (siehe Abbildung \ref{fig:WinkelfarmApp}), in der mehrere Tiere mit ihren Sichtfeldern dargestellt werden können, und erhalten u.~a. folgende Aufgaben (vgl. \protect\hyperlink{ref-Etzold:2019Praxis4}{Etzold, 2019b, S. 8~ff.}):

\begin{enumerate}
\def\labelenumi{\arabic{enumi}.}
\tightlist
\item
  Setze das Schaf an eine Stelle, an der es von der Kuh gesehen wird, aber die Kuh selbst nicht sieht.
\item
  Setze das Schaf an eine Stelle, an der es nicht von der Kuh gesehen wird.
\item
  Das Schaf will die Kuh verwirren. Bewege es an möglichst viele Orte, an denen es von der Kuh gesehen wird.
\item
  Setze das Schaf an eine Stelle, an der es noch gerade so von der Kuh gesehen wird.
\item
  Wo muss das Schaf lang laufen, damit es die gesamte Zeit gerade so von der Kuh gesehen wird?
\end{enumerate}

An Aufgabe 5 kann z.~B. erkundet werde, dass sich das Schaf geradlinig auf der Grenze zwischen Sichtfeld und Nicht-Sichtfeld bewegen muss. In die eine Richtung ist die Bewegung beliebig fortsetzbar, in die andere durch den Kopf der Kuh begrenzt. Eine mathematische Verallgemeinerung dieser Handlung besteht dann in der Identifizierung des Schenkels (Begrenzung) als Strahl (nur in eine Richtung fortsetzbar) mit dem Scheitelpunkt (Kopf der Kuh) als \emph{Quelle} des Winkelfeldes.



\begin{figure}

{\centering \includegraphics[width=0.75\linewidth]{pictures/1-Winkelfarm} 

}

\caption{Screenshot der App Winkel-Farm (\protect\hyperlink{ref-Etzold:2019}{Etzold, 2019a})}\label{fig:WinkelfarmApp}
\end{figure}

Als \emph{Spezifizierung} kann das Finden der Sichtfeld-Situation als charakterisches Beispiel für ein Winkelfeld angesehen werden. Die \emph{Strukturierung} führt zum dargestellten Lernpfad und den konkreten Aufgabenstellung, über die konkrete Handlungen verallgemeinert werden und damit das mathematische Verständnis aufgebaut wird.\index{Vier-Ebenen-Ansatz!konkrete Ebene|(}

\hypertarget{empirische-ebene}{%
\subsection{Empirische Ebene}\label{empirische-ebene}}

Die\index{Vier-Ebenen-Ansatz!empirische Ebene|(} zuvor beschriebene Lernumgebung wurde in mehreren Zyklen erprobt und dabei die Qualität der Handlungen der Schülerinnen und Schüler beobachtet. Ein Ziel bestand darin, dass möglichst verallgemeinerbare Handlungen (wie oben am Beispiel des Schenkels beschrieben) durchgeführt werden.

Es wird erwartet, dass die Repräsentation eines Sichtfeldes von der Draufsicht über eine semintransparent ausgemalte Teilfläche der Ebene noch nicht bekannt ist. Um diese nachzuvollziehen und mit eigenen Erfahrungen in Bezug zu bringen, wird an den Beginn der Unterrichtsstunde ein Bild des Klassenraumes in der Draufsicht präsentiert (siehe Abbildung \ref{fig:Klassenraum}). Dann soll eine Schülerin oder ein Schüler beschreiben, was sie/er alles sieht, ohne den Kopf zu drehen. Dieser Bereich wird auf dem Bild eingezeichnet, so dass die Repräsentation des Sichtfeldes im Folgenden zur Verfügung steht.

\begin{figure}

{\centering \includegraphics[width=0.75\linewidth]{pictures/1-Klassenraum} 

}

\caption{Klassenraum von oben (Foto: Christian Dohrmann)}\label{fig:Klassenraum}
\end{figure}

In der Erprobung konnte beobachtet werden, dass einige Bedienschwierigkeiten mit der Anwendung den Lernfortschritt hemmten. Dies konnte u.~a. dadurch verbessert werden, dass vor die eigentliche Erarbeitung eine freie Erkundungsphase mit der App (siehe Abbildung \ref{fig:WinkelfarmStart}) gesetzt wurde (\protect\hyperlink{ref-Etzold2021}{Etzold, 2021, S. 147, 152}). Durch spezifische Aufgabenstellungen wurden bestimmte Funktionen der App fokussiert:

\begin{figure}

{\centering \includegraphics[width=0.75\linewidth]{pictures/1-WinkelfarmStart} 

}

\caption{Möglicher Startbildschirm für die freie Erkundungphase}\label{fig:WinkelfarmStart}
\end{figure}

\emph{»Das Pferd soll auf dem Steinpflaster stehen, die Frau soll auf dem Pferd sitzen/stehen. Das Pferd guckt in Richtung der grünen Büsche, die Frau hat die Augen zu. Gleichzeitig versteckt sich die Katze unter der Kuh.«}

Die Einführungsphase über das Klassenraumfoto folgt aus der \emph{Spezifizierung} innerhalb der empirischen Ebene. Das Hinzufügen der freien Erkundungsphase ist dagegen der \emph{Strukturierung} der Analyse zuzuordnen.\index{Winkel|)}\index{Vier-Ebenen-Ansatz!empirische Ebene|)}

\hypertarget{verknuxfcpfung-der-ebenen}{%
\subsection{Verknüpfung der Ebenen}\label{verknuxfcpfung-der-ebenen}}

An den Ausführungen ist schon sichtbar geworden, dass sich die Ebenen nicht immer trennen lassen und teilweise gegenseitig beeinflussen. Auch gehen oft Spezifizierung und Strukturierung ineinander über.

Das ist aber gar nicht schlimm, ganz im Gegenteil. Es zeigt wieder einmal, wie wichtig solch ein ganzheitlicher Ansatz ist, so dass eine stoffdidaktische Analyse aus den diversen Sichtpunkten heraus betrachtet werden sollte.

Wichtig ist v.~a., dass Sie sich als Lehrkraft stets darüber im Klaren sind, dass für eine stoffdidaktische Analyse verschiedene Perspektiven verfolgt werden müssen. Sehen Sie den Vier-Ebenen-Ansatz daher auch als Kontrollinstrument, ob Sie an alles gedacht haben, wenn Sie einen Lerngegenstand intensiv analysieren.

\hypertarget{vier-ebenen-nachbereitung}{%
\section{Zum Nachbereiten}\label{vier-ebenen-nachbereitung}}

\begin{enumerate}
\def\labelenumi{\arabic{enumi}.}
\tightlist
\item
  Lesen Sie den Artikel von Hußmann \& Prediger (\protect\hyperlink{ref-Hussmann:2016}{2016}) zum Vier-Ebenen-Ansatz.
\item
  Reflektieren Sie Ihre bisherige Fach- und Fachdidaktikausbildung in Mathematik dahingehend, welche der aufgeworfenen Fragen Sie zu konkreten Themenbereichen (nicht) beantworten könnten.
\end{enumerate}

\hypertarget{hoch-schulmathematik}{%
\chapter{(Hoch-)Schulmathematik}\label{hoch-schulmathematik}}

\begin{quote}
\textbf{Ziele}

\begin{itemize}
\tightlist
\item
  Sie erkennen den Nutzen der Hochschulmathematik bei der Entscheidungsfindung zur Spezifizierung und Strukturierung der Schulmathematik auf der formalen Ebene des Vier-Ebenen-Ansatzes.
\item
  Sie kennen geeignete Quellen zur Beantwortung der Fragen auf der formalen Ebene des Vier-Ebenen-Ansatz
\end{itemize}

\textbf{Material}

\begin{itemize}
\tightlist
\item
  Folien zum Kapitel 2 (\href{files/Stoffdidaktik2023-02-HochSchulmathematik.pdf}{pdf}, \href{files/Stoffdidaktik2023-02-HochSchulmathematik.key}{Keynote})
\end{itemize}
\end{quote}

\hypertarget{doppelte-diskontinuituxe4t}{%
\section{Doppelte Diskontinuität}\label{doppelte-diskontinuituxe4t}}

Auf der \textcolor{formalColor}{formalen Ebene} des Vier-Ebenene-Ansatzes soll der zu betrachtende Lerngegenstand fachmathematisch untersucht werden, um eine erste Auswahl (\emph{Spezifizierung}) und Anordnung (\emph{Strukturierung}) der Lerninhalte zu ermöglichen. Dies ruft natürlich -- auch für Lerngegenstände der Grundschulmathematik -- danach, die im Studium erworbenen hochschulmathematischen Erkenntnisse zu nutzen. Dieser Ruf wird jedoch (scheinbar!) geschmälert durch eine offensichtliche Ungleichheit zwischen Schulmathematik und Hochschulmathematik. Felix Klein beschreibt dieses Phänomen, das sich insbesondere auf die Ausbildung von Lehrkräften auswirkt, bereits im Übergang vom 19. zum 20. Jahrhunderts als \textbf{\emph{doppelte Diskontinuität}}: »Der junge Student sieht sich am Beginn seines Studiums vor Probleme gestellt, die ihn in keinem Punkte mehr an die Dinge erinnern, mit denen er sich auf der Schule beschäftigt hat; natürlich vergißt er daher alle diese Sachen rasch und gründlich. Tritt er aber nach Absolvierung des Studiums ins Lehramt über, so soll er plötzlich eben diese herkömmliche Elementarmathematik schulmäßig unterrichten; da er diese Aufgabe kaum selbständig mit seiner Hochschulmathematik in Zusammenhang bringen kann, so wird er in den meisten Fällen recht bald die althergebrachte Unterrichtstradition aufnehmen, und das Hochschulstudium bleibt ihm nur eine mehr oder minder angenehme Erinnerung, die auf seinen Unterricht keinen Einfluß hat.« (\protect\hyperlink{ref-Klein1967}{Klein, 1967, S. 1})\footnote{Es handelt sich hier um den Nachdruck eines Werke, dessen erste Auflage 1908 erschien.}

Einen Ausweg, dieser doppelten Diskontinität zu entgehen, sah Klein in einer \textbf{\emph{Elementarmathematik vom höheren Standpunkte aus}} (\protect\hyperlink{ref-Klein1925}{Klein, 1925}, \protect\hyperlink{ref-Klein1955}{1955}, \protect\hyperlink{ref-Klein1967}{1967}). Dabei verfolgt er »das Ziel, im Anschluss an umfassende hochschulmathematische Erfahrungen die Schulmathematik in den erworbenen Wissenskanon fachlich einzubetten« (\protect\hyperlink{ref-Danckwerts2013}{Danckwerts, 2013, S. 78}).

Diesen Gedanken fortsetzend kann man auch »gleich am Anfang des Studiums direkt und explizit an die schulmathematischen Vorerfahrungen an{[}knüpfen{]}, bleibt inhaltlich bei diesen und arbeitet einen höheren Standpunkt heraus, der auf die vertiefte Auseinandersetzung mit der Oberstufenmathematik zielt und prinzipiell mit den bis dahin erworbenen (elementar-)mathematischen Mitteln auskommt.« (\protect\hyperlink{ref-Danckwerts2013}{Danckwerts, 2013, S. 78}) Eine solche Entwicklung hat das Projekt \emph{Mathematik Neu Denken} verfolgt, das gut 100 Jahre nach Kleins Publikationen die Lehramtsausbildung im Fach Mathematik weiterentwickeln wollte (\protect\hyperlink{ref-Beutelspacher2012}{Beutelspacher et al., 2012}). Entwickelt wird daraus eine \textbf{\emph{Schulmathematik vom höheren Standpunkt}}, die auf eine fachliche und verstehensorientierte Durchdringung der Schulmathematik zielt, »ohne im vollen Umfang auf das Instrumentarium der kanonischen {[}\ldots{]} {[}Hochschulmathematik{]} zurückgreifen zu müssen« (\protect\hyperlink{ref-Danckwerts2013}{Danckwerts, 2013, S. 87}).

Im Rahmen der Stoffdidaktik-Veranstaltung sollen beide Ansätze aufgegriffen und an konkreten Beispielen versucht werden, die Fragen der \textcolor{formalColor}{formalen Ebene} zu beantworten. Als bedeutsames Bindeglied zwischen Schul- und Hochschulmathematik stellen sich dabeifundamentale Ideen heraus, die auch schon auf die \textcolor{semanticColor}{semantische Ebene} des Vier-Ebenen-Ansatzes zielen -- siehe dazu Abschnitt \ref{fundamentale-ideen}.

\hypertarget{geeignete-quellen}{%
\section{Geeignete Quellen}\label{geeignete-quellen}}

\begin{itemize}
\item
  Neben den Werken von Felix Klein zu Beginn des 20. Jahrhunderts (\protect\hyperlink{ref-Klein1925}{Klein, 1925}, \protect\hyperlink{ref-Klein1955}{1955}, \protect\hyperlink{ref-Klein1967}{1967}) und aktuellen Ansätzen zum Umgang mit der doppelten Diskontinuität in der Lehramtsausbildung (\protect\hyperlink{ref-Ableitinger2013}{Ableitinger et al., 2013}; \protect\hyperlink{ref-Beutelspacher2012}{Beutelspacher et al., 2012}) liefern die in den 1970er Jahren von Hans Freudenthal verfassten Werke zur \emph{Mathematik als pädagogische Aufgabe} (\protect\hyperlink{ref-Freudenthal1973a}{Freudenthal, 1973c}, \protect\hyperlink{ref-Freudenthal:1973}{1973b}; englischsprachig auch digital verfügbar über \protect\hyperlink{ref-Freudenthal1973c}{Freudenthal, 1973a}) Ansätze, Schul- und Hochschulmathematik miteinander in Bezug zu bringen.
\item
  Ebenfalls hilfreich sind größere Nachschlagewerke zur Mathematik, bspw. die \emph{Kleine Enzyklopädie Mathematik} (\protect\hyperlink{ref-Gellert1986}{Gellert et al., 1986}).
\item
  Nicht zu unterschätzen für die fachmathematische Auseinandersetzung sind auch fachdidaktische Quellen, insbesondere zur \textbf{Didaktik der Sachgebiete}. Als digital verfügbare Quellen seien zu erwähnen:

  \begin{itemize}
  \tightlist
  \item
    \emph{Didaktik der Algebra: nach der Vorlage von Hans-Joachim Vollrath} (\protect\hyperlink{ref-Weigand2022}{Weigand et al., 2022})
  \item
    \emph{Didaktik der Geometrie für die Sekundarstufe I} (\protect\hyperlink{ref-Weigand2018}{Weigand et al., 2018})
  \item
    \emph{Didaktik der Analysis. Aspekte und Grundvorstellungen zentraler Begriffe} (\protect\hyperlink{ref-Greefrath2016}{Greefrath et al., 2016})
  \item
    \emph{Didaktik der Stochastik in der Sekundarstufe I} (\protect\hyperlink{ref-Kruger2015}{Krüger et al., 2015})
  \item
    \emph{Didaktik der Analytischen Geometrie und Linearen Algebra: Algebraisch verstehen -- Geometrisch veranschaulichen und anwenden} (\protect\hyperlink{ref-Henn2015}{Henn \& Filler, 2015})
  \item
    \emph{Mathematikunterricht in der Sekundarstufe II. Band 1: Fachdidaktische Grundfragen, Didaktik der Analysis} (\protect\hyperlink{ref-Tietze:2000a}{Tietze et al., 2000a})
  \item
    \emph{Mathematikunterricht in der Sekundarstufe II. Band 2: Didaktik der Analytischen Geometrie und Linearen Algebra} (\protect\hyperlink{ref-Tietze:2000}{Tietze et al., 2000b})
  \item
    \emph{Mathematikunterricht in der Sekundarstufe II. Band 3: Didaktik der Stochastik} (\protect\hyperlink{ref-Tietze:2002}{Tietze et al., 2002})
  \end{itemize}
\item
  Ebenfalls hilfreich für die fachliche Spezifizierung und Strukturierung kann die Darstellung der Fachinhalte in \textbf{Schulbüchern} sein. Hier bietet sich eine vergleichende Analyse mehrerer Schulbücher, auch unterschiedlicher Bundesländer, an.
\item
  Nur gering geeignet für die Spezifizierung und Strukturierung sind die Bildungsstandstandards und Rahmenlehrpläne. Sie bieten -- entsprechend ihrer Funktion -- bereits eine Auswahl der zu unterrichtenden Inhalte und schränken damit die fachdidaktische Diskussion diesbezüglich ein.
\end{itemize}

\hypertarget{beispiel-negative-zahlen}{%
\section{Beispiel Negative Zahlen}\label{beispiel-negative-zahlen}}

Am Beispiel der negativen Zahlen soll dargestellt werden, wie eine Sichtweise vom höheren Standpunkt auf die in der Schule relevante Behandlung dieses Lerngegenstands zur Spezifizierung und Strukturierung helfen kann. Dabei beinhalten die negativen Zahlen sowohl die ganzen Zahlen als auch die rationalen Zahlen.

\hypertarget{natuxfcrliche-zahlen}{%
\subsection{Natürliche Zahlen}\label{natuxfcrliche-zahlen}}

Fachmathematisch können die ganzen Zahlen aus den natürlichen Zahlen generiert werden. Hierzu sollen zunächst die natürlichen Zahlen selbst fachmathematisch eingeordnet werden. Im Prinzip bestehen zwei Sichtweisen, nämlich die Einführung über die \textbf{Peano-Axiome} sowie die Betrachtung \textbf{gleichmächtiger Mengen}.

Aus den Peano-Axiomen (\protect\hyperlink{ref-WikiPeano}{Wikipedia, 2021}) folgt zunächst die Existenz einer Reihenfolge von Zahlen (also Nachfolger der \(0\)), die dann mit \(1\), \(2\), \(3\) usw. bezeichnet werden können.\footnote{Ab 10 wird bei der Bezeichnung jedoch das Stellenwertsystem genutzt -- das geht schon weiter als hier zulässig.} Diese Bezeichnung erlaubt jedoch noch keinerlei Berechnungen, nicht einmal eine Ordnungsrelation ist vorhanden. Es kann also (noch) nicht gesagt werden, dass \(3\) größer ist als \(1\). Vielmehr lässt sich die Situation eher mit einem Alphabet vergleichen\footnote{Ein wesentlicher Unterschied dieses Vergleiches ist, dass das Alphabet endlich ist, die Menge der natürlichen Zahlen jedoch nicht.}, bei dem auch nicht C größer als A ist.

Für eine Ordnungsrelation bedarf es zunächst der Definition der Addition über \(n+0 := n\) und \(n+k' := (n+k)'\) für alle \(n,k\in\mathbb{N}\) mit der (aus den Peano-Axiomen existierenden) Nachfolgerbildung \('\). So gilt etwa mit \(1:=0'\): \({\color{blue} 1}+{\color{red} 1} = {\color{blue} 1}+{\color{red} {0'}} = ({\color{blue} 1}+{\color{red} 0}){\color{red} '} = 1{\color{red} '} =: 2\). So kann nun induktiv jede höhere Additionsaufgabe generiert werden. Darauf aufbauend kann die Ordnungsrelation \(n<m\) über die Existenz eines \(k\in\mathbb{N}\backslash\{0\}\) mit \(m = n+k\) definiert werden. Die Subtraktion \(m-n = k\) ist nun wiederum über die Umkehroperation \(n+k = m\) definierbar, sofern \(m\geq n\).

Die Gleichmächtigkeit (z.~B. endlicher) Mengen \(M\) und \(N\) wird über die Existenz einer Bijektion zwischen diesen beiden Mengen definiert, siehe Abbildung \ref{fig:Bijektion}.

\begin{figure}

{\centering \includegraphics[width=0.75\linewidth]{pictures/2-Bijektion} 

}

\caption{Gleichmächtigkeit von Mengen}\label{fig:Bijektion}
\end{figure}

Diese Relation ist eine Äquivalenzrelation, also symmetrisch, reflexiv und transitiv. Damit können Äquivalenzklassen gebildet werden, die die Mächtigkeit der Menge angeben. Statt \([4]\) als Bezeichner für die Äquivalenzklasse von vierelementigen Mengen kann dann auch kurz die Zahl \(4\) genutzt werden. Die Addition \(n+k\) entspricht dann der Mächtigkeit der Vereinigungsmenge von \([n]\) und \([k]\), vgl. Abbildung \ref{fig:Vereinigung}.

\begin{figure}

{\centering \includegraphics[width=0.9\linewidth]{pictures/2-Vereinigung} 

}

\caption{Additionsergebnis als Äquivalenzklasse der Vereinigungsmenge}\label{fig:Vereinigung}
\end{figure}

\hypertarget{ganze-zahlen}{%
\subsection{Ganze Zahlen}\label{ganze-zahlen}}

Da innerhalb der natürlichen Zahlen noch nicht beliebig subtrahiert werden darf, stehen auch keine negativen Zahlen als Ergebnisse zur Verfügung. Um dennoch das Ergebnis bspw. der Aufgabe \(2-7\) »definieren« zu können, bietet sich erneut eine \textbf{Äquivalenzrelation} über die »Differenzengleichheit« an. Konkret lässt sich für \(k,l,m,n\in\mathbb{N}\) sagen:
\((k,l)\sim (m,n):\Leftrightarrow k+n=l+m\)
Das heißt z.~B., dass die Zahlenpaare \((2,7)\), \((0,5)\) und \((4,9)\) in Relation zueinander stehen, weil sie dieselbe »Differenz« haben (obwohl es die Differenz formal noch nicht gibt). Dies ermöglicht nun die Einführung des Bezeichners \(-5\) für die Äquivalenzklasse \([(0,5)]\).

Das Vorgehen ist \emph{verträglich} mit den bisherigen Regeln in \(\mathbb{N}\), d.~h. es führt nicht zu Widersprüchen, wenn etwa das Zahlenpaar \((7,4)\) betrachtet wird mit dem Repräsentanten-Bezeichner \(3\). Die Menge aller Äquivalenzklassen (bzw. deren Kurzbezeichner) ist nun \(\mathbb{Z}\).

Die Addition und Subtraktion zweier Zahlenpaare sind nun definierbar:

\begin{align}
(k, l) + (m, n) := (k + m, l + n)\\
(k, l) − (m, n) := (k, l) + (n, m)
\end{align}

Als Alternative bietet sich ein \textbf{axiomatisches Vorgehen} an, also dass die ganzen Zahlen mit der Addition als abelsche Gruppe definiert werden -- bedeutsam ist hier insbesondere die Existenz eines Inversen zu jeder Zahl.

\hypertarget{permanenzprinzip-und-permanenzreihen}{%
\subsubsection{Permanenzprinzip und Permanenzreihen}\label{permanenzprinzip-und-permanenzreihen}}

Wie bereits erwähnt, führen die neu eingeführte Addition und Subtraktion in \(\mathbb{Z}\) nicht zu Konflikten mit den bisherigen analogen Operationen in \(\mathbb{N}\). Dies wird über das \textbf{\emph{Permanenzprinzip}} gefordert, nach dem neue Theorien (z.~B. das Rechnen mit negativen Zahlen) soweit wie möglich verträglich sein müssen mit bisherigen Theorien (z.~B. das Rechnen mit positiven Zahlen).

Sichtbar gemacht werden kann dieses Prinzip und die Herleitung gewisser Rechenregeln über \textbf{\emph{Permanenzreihen}}. Dies ist insbesondere dann hilfreich, wenn für bestimmte Rechenoperationen keine geeigneten realen Veranschaulichungen existieren. Ein typisches Beispiel hierfür ist die Multiplikation zweier negativer Zahlen. Kann die Multiplikation einer natürlichen Zahl (erster Summand) mit einer negativen Zahl (zweiter Summand) außermathematisch noch als mehrfache Verschuldung aufgefasst und die Vertauschung von erstem und zweitem Summanden über das Kommutativgesetz innermathematisch erklärt werden, bietet die Multiplikation zweier negativer Zahlen keine so naheliegende Interpretation. Abbildung \ref{fig:Permanenz} stellt eine Permanenzreihe dar, anhand derer die Rechnung \((-3)\cdot (-5)\) erklärt werden kann.

\begin{figure}

{\centering \includegraphics[width=0.3\linewidth]{pictures/2-Permanenz} 

}

\caption{Permanenzreihe zur Multiplikation zweier negativer Zahlen}\label{fig:Permanenz}
\end{figure}

\hypertarget{ableitungen-fuxfcr-den-lernpfad}{%
\subsubsection{Ableitungen für den Lernpfad}\label{ableitungen-fuxfcr-den-lernpfad}}

Aus all den bisherigen Überlegungen auf der formalen Ebene lassen sich für den Unterricht zentrale Fachinhalte ableiten:

\textcolor{formalColor}{Ganze Zahlen können über Zahlenpaare aus den natürlichen Zahlen oder als »Gegenzahlen« der natürlichen Zahlen entwickelt werden.}

\begin{itemize}
\tightlist
\item
  \textcolor{formalColor}{Natürliche Zahlen sind als Teilmenge in die ganzen Zahlen eingebettet.}
\item
  \textcolor{formalColor}{Die Subtraktion natürlicher Zahlen $m-n$ mit $n > m$ ist nun lösbar.}
\item
  \textcolor{formalColor}{Die Rechenregeln werden erweitert, wobei die bekannten weiter gelten. Die wird über das Permanenzprinzip begleitet, bei der Herleitung von Rechenregeln bietet sich die Nutzung von Permanenzreihen an.}
\end{itemize}

\hypertarget{rationale-zahlen}{%
\subsection{Rationale Zahlen}\label{rationale-zahlen}}

In fachlich analoger Weise lassen sich auch die rationalen Zahlen über Äquivalenzrelationen einführen. Dann fordert die »Quotientengleichheit«, dass für \(k,l,m,n\in\mathbb{N}\) mit \(l,n\neq 0\) gilt: \((k,l)\sim (m,n):\Leftrightarrow k\cdot n=l\cdot m\). Die Äquivalenzklasse \([(1,2)]\) kann dann mit \(\frac{1}{2}\) bezeichnet werden. Im Gegensatz zu den ganzen Zahlen ist es bei den rationalen Zahlen durchaus üblich, für dieselbe Zahl unterschiedliche Bezeichner zu verwenden, wie \(\frac{1}{2}\) oder \(\frac{5}{10}\). Fachmathematisch ist dies jedoch nicht relevant, also auch keine Diskussion auf der formalen Ebene (jedoch auf späteren Ebenen).

Aus Sicht der formalen Ebene lässt sich daher auch nicht ableiten, ob im Mathematikunterricht nach den natürlichen Zahlen zunächst die rationalen Zahlen (wie z.~B. in Deutschland) oder erst die ganzen Zahlen (wie z.~B. in Australien) eingeführt werden sollten. Innerhalb eines Zahlbereichs bietet jedoch die fachlogische Struktur Ansatzpunkte zur Gestaltung des Lernpfads, wie bei den negativen Zahlen dargestellt.

\hypertarget{hochschulmathematik-nachbereitung}{%
\section{Zum Nachbereiten}\label{hochschulmathematik-nachbereitung}}

\begin{enumerate}
\def\labelenumi{\arabic{enumi}.}
\tightlist
\item
  Nutzen Sie verschiedene fachmathematische und fachdidaktische Quellen sowie Schulbücher, um fachlich zu klären, was »Terme« und »Gleichungen« sind.
\item
  Nutzen Sie Permanenzreihen, um weitere Rechengesetze nachzuvollziehen, z.~B. dass \(a^0 = 1\) (\(a\neq 0\)) und \(0^0\) nicht definiert ist.
\end{enumerate}

\hypertarget{mathematik-strukturieren}{%
\chapter{Mathematik strukturieren}\label{mathematik-strukturieren}}

\begin{quote}
\textbf{Ziele}

\begin{itemize}
\tightlist
\item
  Sie kennen verschiedene Möglichkeiten, Mathematik zu strukturieren.
\item
  Sie können beschreiben, woher die verschiedenen Strukturierungsmöglichkeiten kommen.
\item
  Sie kennen Beispiele für fundamentale Ideen der Mathematik.
\item
  Sie können bei einzelnen Lerngegenständen den Zusammenhang zu zugehörigen fundamentalen Ideen herstellen.
\end{itemize}

\textbf{Material}

\begin{itemize}
\tightlist
\item
  Folien zum Kapitel 3 (\href{files/Stoffdidaktik2023-03-MathematikStrukturieren.pdf}{pdf}, \href{files/Stoffdidaktik2023-03-MathematikStrukturieren.key}{Keynote})
\end{itemize}
\end{quote}

\hypertarget{sachgebiete}{%
\section{Sachgebiete}\label{sachgebiete}}

Für die Fachwissenschaft Mathematik haben sich historisch verschiedene Unterdisziplinen entwickelt, die als Sachgebiete der Mathematik bezeichnet werden können. Schulrelevante Gebiete sind hierbei:

\begin{itemize}
\tightlist
\item
  Arithemtik
\item
  Algebra
\item
  Geometrie
\item
  Analysis
\item
  Stochastik
\item
  Lineare Algebra / Analytische Geometrie
\end{itemize}

Auch heute bilden sich diese und weitere Sachgebiete (z.~B. Numerik) in den Strukturen von universitären Lehrveranstaltungen, Forschungsrichtungen und nicht zuletzt der Strukturierung einzelner Lehrpläne der Schulen ab.

Für eine Vertiefung mit der Didaktik der Sachgebiete eignen sich u.~a. die in Abschnitt \ref{geeignete-quellen} dargestellten Quellen. Weiterhin werden Sie im Masterstudium im Modul \emph{Ausgewählte Themen der Mathematikdidaktik}\footnote{siehe Modulbeschreibung zum Modul \href{https://puls.uni-potsdam.de/qisserver/rds?state=verpublish\&status=init\&vmfile=no\&moduleCall=modulansicht\&publishConfFile=modulverwaltung\&publishSubDir=up/modulbearbeiter\&\&modul.modul_id=3186\&menuid=\&topitem=Modulbeschreibung\&subitem=}{MAT-LS-D3 bei PULS}} die Möglichkeit haben, sich mit der Didaktik einzelner Sachgebiete näher auseinanderzusetzen.

\hypertarget{leitideen}{%
\section{Leitideen}\label{leitideen}}

Die Strukturierung mathematischer Inhalte in Leitideen ist seit Anfang der 2000er Jahre im deutschen Bildungswesen etabliert, als die KMK\footnote{Mehr zur Kultusministerkonferenz (KMK) und ihrer eigentlichen Bezeichnungen siehe Wikipedia (\protect\hyperlink{ref-dewiki:228417777}{2022}).} \textbf{Bildungsstandards} für den Mittleren Schulabschluss (2004), den Primarbereich (2005) und später auch die Allgemeine Hochschulreife (2012) herausgebracht hat. Darauf aufbauend wurden in den meisten Bundesländern die Lehrpläne angepasst.
Zwischenzeitlich wurden die Bildungsstandards für den Primarbereich sowie den Ersten und Mittleren Schulabschluss überarbeitet, für die gymnasiale Oberstufe gelten während der gerade laufenden Überarbeitung noch die von 2012. (\protect\hyperlink{ref-KMK:2012}{Sekretariat der Ständigen Konferenz der Kultusminister der Länder in der Bundesrepublik Deutschland, 2012}, \protect\hyperlink{ref-SekretariatderStandigenKonferenzderKultusministerderLanderinderBundesrepublikDeutschland2022a}{2022b}, \protect\hyperlink{ref-SekretariatderStandigenKonferenzderKultusministerderLanderinderBundesrepublikDeutschland2022}{2022a}).
In Brandenburg spiegeln sich die Bildungsstandards in den \textbf{Rahmenlehrplänen} für die Jahrgangsstufen 1~--~10 und die Gymnasiale Oberstufe wider (\protect\hyperlink{ref-MinisteriumfurBildungJugendundSportdesLandesBrandenburg2022}{Ministerium für Bildung, 2022}, \protect\hyperlink{ref-MinisteriumfuerBildungJugendundSportdesLandesBrandenburg2023}{2023}).

Im Laufe der letzten 20 Jahre haben sich für die Leitideen teils verschiedene Bezeichnungen ergeben. In den aktuellen Bildungsstandards des Ersten und Mittleren Schulabschlusses (\protect\hyperlink{ref-SekretariatderStandigenKonferenzderKultusministerderLanderinderBundesrepublikDeutschland2022}{Sekretariat der Ständigen Konferenz der Kultusminister der Länder in der Bundesrepublik Deutschland, 2022a}) werden verwendet:

\begin{itemize}
\tightlist
\item
  Leitidee Zahl und Operation
\item
  Leitidee Größen und Messen
\item
  Leitidee Strukturen und funktionaler Zusammenhang
\item
  Leitidee Raum und Form
\item
  Leitidee Daten und Zufall
\end{itemize}

Die Leitidee Zahl und Operation beispielweise »umfasst sinntragende Vorstellungen und Darstellungen von Zahlen und Operationen sowie die Nutzung von Rechengesetzen und Kontrollverfahren. Dazu gehören die sachgerechte Nutzung von Prozent- und Zinsrechnung ebenso wie kombinatorische Überlegungen und Verfahren, denen Algorithmen zu Grunde liegen.« (\protect\hyperlink{ref-SekretariatderStandigenKonferenzderKultusministerderLanderinderBundesrepublikDeutschland2022}{Sekretariat der Ständigen Konferenz der Kultusminister der Länder in der Bundesrepublik Deutschland, 2022a, S. 15}). Weiterhin werden diese Kompetenzen an spezifischen Fachinhalten konkretisiert, etwa: »Die Schülerinnen und Schüler {[}\ldots{]} • untersuchen Zahlen nach ihren Faktoren, in einfachen Fällen ohne digitale Mathematikwerkzeuge, • stellen Zahlen der Situation angemessen dar, z.B. unter anderem in Zehnerpotenzschreibweise, • rechnen mit natürlichen, ganzen und rationalen Zahlen, die im täglichen Leben vorkommen, sowohl zur Kontrolle als auch im Kopf und erklären die Bedeutung der Rechenoperationen {[}\ldots{]}« (\protect\hyperlink{ref-SekretariatderStandigenKonferenzderKultusministerderLanderinderBundesrepublikDeutschland2022}{Sekretariat der Ständigen Konferenz der Kultusminister der Länder in der Bundesrepublik Deutschland, 2022a, S. 15})

Die Leitideen werden in den Bildungsstandards als \textbf{inhaltsbezogene Kompetenzen} beschrieben, die mit Abschluss des ersten bzw. mittleren Schulabschlusses zu erreichen sind. Es handelt sich dabei um \textbf{Regelstandards}, also Kompetenzen, die »Schülerinnen und Schüler im Durchschnitt in einem Fach erreichen sollen« (\protect\hyperlink{ref-SekretariatderStandigenKonferenzderKultusministerderLanderinderBundesrepublikDeutschland2022}{Sekretariat der Ständigen Konferenz der Kultusminister der Länder in der Bundesrepublik Deutschland, 2022a, S. 2}). Diese sind abzugrenzen gegenüber \textbf{Basiskompetenzen} als »mathematisch zentrale, instrumentell bedeutsame und geradezu grundlegende Konzepte und Verfahren, die für die mathematische Kompetenzentwicklung unverzichtbar sind« (vgl. \url{https://pikas-mi.dzlm.de/node/92}). Für Basiskompetenzen gibt es derzeit in Deutschland keine politischen Dokumente wie Rahmenlehrpläne oder Bildungsstandards

\hypertarget{arten-mathematischen-wissens}{%
\section{Arten mathematischen Wissens}\label{arten-mathematischen-wissens}}

Eine weitere Strukturierung mathematischen Wissens kann darin bestehen, den Blick darauf zu lenken, wie dieses Wissen angeeignet wird. So können ähnliche Aneigungsprozesse Motivation bieten, das Wissen entsprechend zu strukturieren und dies dann für die Gestaltung von Lehr-Lern-Prozessen nutzbar zu machen. Etabliert hat sich hierfür eine Unterscheidung in drei Arten mathematischen Wissens (vgl. \protect\hyperlink{ref-Vollrath2012}{Vollrath \& Roth, 2012, S. 45}~f.):

\begin{itemize}
\item
  \textbf{Begriffe.} Diese bilden das Grundgerüst der Mathematik und belegen Objekte gleicher Eigenschaft mit einem gemeinsamen Bezeichner. Neben der Begriffsfestlegung, i.~d.~R. über eine Definition, ist der Einsatz geeignter Beispiele und Gegenbeispiele essentiell beim Aufbau eines Begriffsverständnisses.
\item
  \textbf{Sachverhalte.} Diese beschreiben Eigenschaften von Begriffen und ihre Beziehungen zueinander. Klassischerweise gehören hierzu mathematische Sätze inkl. ihrer Beweise, aber auch präformale Begründungen. Je nach Akzentuierung sprechen andere Quellen statt von Sachverhalten auch von Sätzen oder von Zusammenhängen.
\item
  \textbf{Verfahren.} Diese bestimmen, wie bestimmte Aufgaben zu lösen sind, z. B. schriftliche Rechenverfahren, Lösungsverfahren von Gleichungen und Gleichungssystemen.
\end{itemize}

Einige Autoren zählen zu den Verfahren auch heuristische Strategien zum Problemlösen oder die Anwendung des Permanenzprinzips (vgl. \protect\hyperlink{ref-Steinhofel1988}{Steinhöfel et al., 1988, S. 23}) Vollrath \& Roth (\protect\hyperlink{ref-Vollrath2012}{2012, S. 46}~ff.) ergänzen dagegen drei Wissensarten noch:

\begin{itemize}
\tightlist
\item
  \textbf{Metamathematisches Wissen.} Darunter ist zu verstehen, \emph{wie} Mathematik betrieben wird, also bspw. welche Möglichkeiten es gibt, ein mathematisches Problem zu lösen oder eine Sachsituation mathematisch zu modellieren.
\end{itemize}

In Abschnitt \ref{begriffe-sachverhalte-verfahren} wird näher darauf eingegangen, wie Lernprozesse bei der Ausbildung der jeweiligen Wissensarten gestaltet werden können.

\hypertarget{fundamentale-ideen}{%
\section{Fundamentale Ideen}\label{fundamentale-ideen}}

\hypertarget{fundamentale-ideen-begriffsklaerung}{%
\subsection{Begriffsklärung}\label{fundamentale-ideen-begriffsklaerung}}

Die Entwicklung Fundamentaler Ideen beruft sich auf Bruners Annahme, dass »jedes Kind {[}\ldots{]} auf jeder Entwicklungsstufe jeder Lehrgegenstand in einer intellektuell ehrlichen Form erfolgreich gelehrt werden« kann (vgl. \protect\hyperlink{ref-Bruner:1976}{Bruner, 1976, S. 77}). Voraussetzung dafür ist, dass die \emph{Struktur} eines Inhaltsbereichs in einer Art und Weise präsentiert wird, dass sie dem Kind zugänglich wird. Diese \emph{hinter den Dingen} liegende Struktur hebt sich vom konkreten Inhaltsbereich ab, ist allgemeinerer Natur und kann daher über \emph{Fundamentale Ideen} beschrieben werden.

Ziel der Orientierung des Unterrichtens an Fundamentalen Ideen besteht v.~a. darin, die (oftmals) isolierten Stoffelemente einzuordnen und in einem größeren Ganzen zu sehen. Im Umkehrschluss heißt dies aber auch, dass die Auswahl des konkreten Stoffes daran orientiert sein muss, wie dieser dazu beitragen kann, den dahinter liegenden mathematischen Kern und die zugehörigen Fundamentalen Ideen zu vertreten.

Die dazu seit den 1960er Jahren in Gang gesetzte Forschung führte zu vielfältigen Vorschlägen Fundamentaler Ideen der Mathematik -- jedoch bisher nicht zu einem allgemeingültigen Katalog. Dieser Vielfalt in den Formulierungen und Kategorisierungen kann begegnet werden, indem Fundamentale Ideen über Eigenschaften charakterisiert werden. Im Rahmen dieser Veranstaltung wird folgende Definition genutzt, zitiert aus Schwill (\protect\hyperlink{ref-Schwill:1994}{1994}).

\begin{definition}[Fundamentale Idee]
\protect\hypertarget{def:FundamentaleIdee}{}\label{def:FundamentaleIdee}

Eine \textbf{Fundamentale Idee}\index{Fundamentale Idee|textbf} bzgl. eines Gegenstandsbereichs (Wissenschaft, Teilgebiet) ist ein \textbf{Denk-, Handlungs-, Beschreibungs- oder Erklärungsschema}, das

\begin{enumerate}
\def\labelenumi{\arabic{enumi}.}
\tightlist
\item
  in verschiedenen Gebieten des Bereichs vielfältig anwendbar oder erkennbar ist (\textbf{Horizontalkriterium}),\index{Fundamentale Idee!Horizontalkriterium|textbf}
\item
  auf jedem intellektuellen Niveau aufgezeigt und vermittelt werden kann (\textbf{Vertikalkriterium}),\index{Fundamentale Idee!Vertikalkriterium|textbf}
\item
  in der historischen Entwicklung des Bereichs deutlich wahrnehmbar ist und längerfristig relevant bleibt (\textbf{Zeitkriterium}),\index{Fundamentale Idee!Zeitkriterium|textbf}
\item
  einen Bezug zu Sprache und Denken des Alltags und der Lebenswelt besitzt (\textbf{Sinnkriterium}).\index{Fundamentale Idee!Sinnkriterium|textbf}
\end{enumerate}

\end{definition}

\begin{quote}
\textbf{Überblick zur historischen Entwicklung Fundamentaler Ideen}

\begin{itemize}
\tightlist
\item
  von der Bank (\protect\hyperlink{ref-Bank:2016}{2016, 37~ff.}): \emph{Fundamentale Ideen der Mathematik: Weiterentwicklung einer Theorie zu deren unterrichtspraktischer Nutzung}
\end{itemize}
\end{quote}

Fundamentale Ideen haben zwar ihren Ursprung in der Fachstruktur, aber sie »sind nicht Elemente der Wissenschaft an sich, sondern Produkte unseres Verstandes, die wir der Wissenschaft aufprägen. Folglich können sie nur relativ zum Menschen objektiviert werden« (\protect\hyperlink{ref-Schubert:2011}{Schubert \& Schwill, 2011, S. 62}).

Für Ihre stoffdidaktische Analyse können Fundamentale Ideen insbesondere hilfreich für die \textbf{Dekonstruktion} des Fachwissens und anschließende \textbf{Rekonstruktion} des Schulwissens sein.

Wenn sie also beispielsweise eine stoffdidaktische Analyse zur Flächeninhaltsberechnung durchführen, setzen Sie sich mit der Fundamentalen Idee des \emph{Messens}\index{Messen} auseinander. Dabei verstehen Sie Messvorgänge als Vergleiche zu einem Standardmaß (z.~B. Kästchen auszählen), erkennen Zerlegungs- und Ergänzungsgleichheit als notwendige Prinzipien zur präziseren Beschreibung, sehen Dreiecke als bedeutsame Basisfiguren für Flächeninhaltsberechnungen an und haben den Blick für die Integralrechnung als verallgemeinerbare Methode zur Flächeninhaltsbestimmung krummliniger Figuren (vgl. \protect\hyperlink{ref-Vohns:2000}{Vohns, 2000, 98~ff.}). Sie \emph{dekonstruieren} (zerlegen) damit Ihr eigenes mathematisches Fachwissen.

Nun sind Sie in der Lage, das Wissen zur Flächeninhaltsberechnung für Schülerinnen und Schüler neu aufzubauen, also zu \emph{rekonstruieren} und (unter Hinzunahme der Betrachtung von Grundvorstellungen und den restlichen Ebenen des Vier-Ebenen-Ansatzes) einen Lernpfad zu entwickeln. Im Zusammenhang mit der Integralrechnung kann dies z.~B. heißen, dass Sie parallel zum Bilden von Ober- und Untersummen noch einmal eine krummlinig begrenzte Fläche durch Kästchen auszählen lassen -- ggf. mit unterschiedlicher Feinheit und einer Abschätzung nach oben und nach unten. Die Fundamentalen Ideen haben für Sie damit auch eine \emph{ordnende Funktion} des Unterrichtsstoffes.

\begin{figure}

{\centering \includegraphics[width=0.75\linewidth]{pictures/3-Flaeche} 

}

\caption{Flächeninhaltsbestimmung}\label{fig:Flaeche}
\end{figure}

\hypertarget{auswahl-fundamentaler-ideen}{%
\subsection{Auswahl fundamentaler Ideen}\label{auswahl-fundamentaler-ideen}}

Das Fehlen eines allgemeingültigen Katalogs sollte nicht davon abhalten, bestehende Auflistungen und Strukturierungen Fundamentaler Ideen zu betrachten. von der Bank (\protect\hyperlink{ref-vonderBank:2013}{2013, S. 103}) und Lambert (\protect\hyperlink{ref-Lambert:2012}{2012}) diskutieren eine Kategorisierung fundamentaler Ideen in drei Bereiche:

\begin{itemize}
\item
  \textbf{Inhaltsideen} beziehen sich auf konkrete Inhaltsbereiche der Mathematik, die die Kriterien Fundamentaler Ideen erfüllen können. Nicht ganz zufällig spiegeln diese sich in den Leitideen der Bildungsstandards wider (siehe Abschnitt \ref{leitideen}).
\item
  \textbf{Schnittstellenideen} haben die Eigenschaft, dass durch sie die »Mathe(matik) wirkt« und »auch für andere Fächer in ihrer je spezifischen Weise relevant sind« (\protect\hyperlink{ref-Lambert:2012}{Lambert, 2012}). Damit korrelieren sie mit den prozessbezogenen Kompetenzen der Bildungsstandards.
\item
  \textbf{Tätigkeitsideen} beziehen sich insbesondere auf innermathematische Tätigkeiten, die sich über verschiedene Inhaltsbereiche hinweg zeigen. Lambert (\protect\hyperlink{ref-Lambert:2012}{2012}) betont, dass es diese (über die Bildungsstandards hinaus) ebenfalls zu beachten gilt, wenn man einen reichhaltigen Mathematikunterricht bewirken möchte.
\end{itemize}

Beispiele derartiger Tätigkeitsideen sind:

\begin{itemize}
\tightlist
\item
  Approximierung
\item
  Optimierung
\item
  Linearität/Linearisierung
\item
  Symmetrie
\item
  Invarianz
\item
  Rekursion
\item
  Vernetzung
\item
  Ordnen
\item
  Strukturierung
\item
  Formalisierung
\item
  Exaktifizierung
\item
  Verallgemeinern
\item
  Idealisieren
\end{itemize}

Im Rahmen des Projektmoduls \emph{Erweitertes Fachwissen für den schulischen Kontext in Mathematik}\footnote{siehe Modulbeschreibung zum Modul \href{https://puls.uni-potsdam.de/qisserver/rds?state=verpublish\&status=init\&vmfile=no\&moduleCall=modulansicht\&publishConfFile=modulverwaltung\&publishSubDir=up/modulbearbeiter\&\&modul.modul_id=3156\&menuid=\&topitem=Modulbeschreibung\&subitem=}{MAT-LS-7 bei PULS}} werden Sie insbesondere Bezüge zwischen Schul- und Hochschulmathematik auf Basis Fundamentaler Ideen herstellen, wofür die Inhalts- und Tätigkeitsideen von hoher Relevanz sind.

\hypertarget{beispiel-linearitaet}{%
\subsection{Beispiel Linearität}\label{beispiel-linearitaet}}

\hypertarget{horizontal--und-vertikalkriterium}{%
\subsubsection{Horizontal- und Vertikalkriterium}\label{horizontal--und-vertikalkriterium}}

Linearität\index{Linearität|(}\index{Fundamentale Idee!Horizontalkriterium|(}\index{Fundamentale Idee!Vertikalkriterium|(} ist ein wesentliches Konzept über die gesamte Schullaufbahn hinweg (und darüber hinaus). Dies spiegelt sich in vielfältigen Themenbereichen wider, die sowohl die Breite (\emph{Horizontalkriterium}) als auch Tiefe (\emph{Vertikalkriterium}) von Linearität und (später) auch Linearisierung zeigen. Dieser Abschnitt orientiert sich an den Darstellungen von Danckwerts (\protect\hyperlink{ref-Danckwerts:1988}{1988}).

\begin{itemize}
\tightlist
\item
  Linearität als Phänomen tritt schon im Geometrieunterricht der Grundschule mit \textbf{Geraden} als essentielle geometrische Objekte auf. In der euklidischen Geometrie sind Geraden neben Punkten die Basisobjekte eines axiomatischen Aufbaus.
\item
  Das \textbf{Distributivgesetz} \(a\cdot (b+c) = a\cdot b + a\cdot c\), das ebenfalls bereits in der Grundschule behandelt wird, beschreibt einen linearen Vorgang und bietet die Grundlage für die halbschriftliche Multiplikation. Über die Schulmathematik hinaus dient es z.~B. als eines der Vektorraumaxiome (Skalarmultiplikation).
\item
  Das Bestimmen eines \textbf{Rechteckflächeninhalts} ist ein linearer Vorgang: Ein Rechteck, das doppelt so breit ist, hat (bei gleicher Höhe) einen doppelt so großen Flächeninhalt. Betrachtet man diese Eigenschaft nicht als Phänomen, sondern als Forderung an eine Flächeninhaltsformel, so kann aus den Bedingungen \(A(a_1+a_2,b) = A(a_1,b) + A(a_2,b)\) und \(A(a,b_1+b_2) = A(a,b_1)+A(a,b_2)\) sowie der Stetigkeit in \(\mathbb{R}^+\) die Formel \(A(a,b) = a\cdot b\) abgeleitet werden.
\item
  Lineare Zuordnungen der Art \(f(x+y) = f(x)+f(y)\) werden zu Beginn der Sekundarstufe I als \textbf{proportionale Zuordnungen} behandelt. Dies wird fortgeführt bei \textbf{linearen Funktionen} der Art \(f(x) = mx+n\), in der Fachmathematik als affin-lineare Abbildungen bezeichnet.
\item
  \textbf{Lineare Gleichungen und Gleichungssysteme} sind ebenfalls bedeutsamer Bestandteil des Mathematikunterrichts. Überhaupt baut die gesamte \textbf{Lineare Algebra} auf lineare und affin-lineare Abbildungen auf.
\item
  Die \textbf{Strahlensätze} beschreiben ebenfalls ein lineares Verhalten: Geradenabschnitte in \(c\)-facher Entfernung sind \(c\) mal so lang.
\end{itemize}

\begin{figure}

{\centering \includegraphics[width=0.5\linewidth]{pictures/3-Strahlensatz} 

}

\caption{Strahlensatzfigur}\label{fig:Strahlensatz}
\end{figure}

\begin{itemize}
\tightlist
\item
  Beim \textbf{Ableitungsbegriff} ist eine wesentliche Vorstellung, dass die Funktion in der Umgebung der zu betrachtenden Stelle linearisiert wird. Insbesondere bei höherdimensionalen Funktionen wird der Linearisierungsansatz weiterverfolgt. Die ebenfalls vorherrschende Tangentenvorstellung ist auf mehr als drei Dimensionen nicht mehr anschaulich übertragbar -- der Linearisierungsansatz weist hier aufgrund seiner algebraischen Beschreibung die bessere Verallgemeinerbarkeit auf.
\item
  Eng an den Linearisierungsansatz angelehnt ist die \textbf{lineare Approximation} von Funktionen (z.~B. \(\sin(x)\approx x\) für \(x\approx 0\)). Die führt sich in der Hochschulmathematik fort, beispielsweise bei Taylor-Reihen.
\item
  Das Bedürfnis der Linearisierung, insbesondere aus der Physik heraus, zeigt sich auch bei der Nutzung \textbf{spezifisch skalierter Diagrammachsen}, z.~B. von Logarithmuspapier. Wegen der Äquivalenz von \(y = c\cdot a^x\) und \(\ln y = (\ln a )\cdot x + \ln c\) lassen sich beliebige Exponentialfunktionen auf Logarithmuspapier als lineare Funktionen darstellen.
\item
  Verschiedene Näherungsverfahren, wie das \textbf{Newton-Verfahren}, bedienen sich ebenfalls der Linearisierung.
\end{itemize}

An dieser Stelle sei darauf hingewiesen, dass Linearität derart fundamental ist, dass selbst nicht-lineare Zusammenhänge häufig fälschlicherweise als linear angenommen werden. Dies zeigt sich zum Beispiel an den Fehlannahmen \((x+y)^2 \overset{?!}{=} x^2+y^2\), \(\sqrt{x+y} \overset{?!}{=} \sqrt{x}+\sqrt{y}\) oder \(\sin(x+y) \overset{?!}{=} \sin(x)+\sin(y)\). Derartige Fehler können Sie als Lehrkraft besser einordnen (und korrigieren), wenn Sie sich der Fundamentalen Idee \emph{Linearität} (die hier eben \emph{nicht} gilt) bewusst sind. Insbesondere spricht dies auch für ein Explizitmachen der Fundamentalen Idee Ihren Schülerinnen und Schülern gegenüber, so dass Sie derartigen Fehlern nicht nur mit Gegenbeispielen entgegen treten können, sondern auch eine strukturelle Einordnung sichtbar machen können.

Gerade wegen der genannten Fehlannahmen und der für die Schülerinnen und Schüler i.~d.~R. nicht in Zusammenhang gebrachten Dualität aus \emph{geradlinig} und \emph{additiv und homogen} sehen Tietze et al. (\protect\hyperlink{ref-Tietze:2002}{2002, S. 39}) die Linearität dagegen nicht als eine im Mathematikunterricht etablierte Fundamentale Idee, »die die Schüler erkennen und die ihr Denken ordnet und anregt«.\index{Fundamentale Idee!Horizontalkriterium|)}\index{Fundamentale Idee!Vertikalkriterium|)}

\hypertarget{zeit--und-sinnkriterium}{%
\subsubsection{Zeit- und Sinnkriterium}\label{zeit--und-sinnkriterium}}

Linearität\index{Fundamentale Idee!Zeitkriterium|(}\index{Fundamentale Idee!Sinnkriterium|(} zeigt sich auch in der historischen Entwicklung der Mathematik als eine prägende Leitlinie, womit sie das \emph{Zeitkriterium} Fundamentaler Ideen erfüllt. In der Linearen Algebra sei beispielsweise das Lösen linearer Gleichungssysteme im 18. Jahrhundert bis hin zum Gauß-Algorithmus im 19. Jahrhundert oder die Darstellung linearer Vorgänge mit Matrizen im 17./18. Jahrhundert erwähnt (vgl. \protect\hyperlink{ref-Tietze:2000}{Tietze et al., 2000b, 73~ff.}). In der Analysis spiegelt sich die Linearität bzw. Linearisierung in der gesamten Differenzialrechnung wider, von der Interpolation nach der Jahrtausendwende über Taylors \emph{Linear perspective} von 1715 (vgl. \protect\hyperlink{ref-Bruckler:2018}{Brückler, 2018, 39,119}) bis in die Gegenwart der linearen Modellierung nichtlinearer Zusammenhänge.

\begin{quote}
\textbf{Historische Originalausgabe}

Taylor (\protect\hyperlink{ref-Taylor:1715}{1715}): \emph{Linear perspective}
\end{quote}

Auch Alltagssituationen bzw. die Alltagssprache ist von Linearität geprägt. Beispielsweise treten proportionale Zuordnungen unmittelbar beim Einkaufen auf, wenn Waren abgewogen und der Preis bestimmt wird. Auch reale Messvorgänge, wie z.~B. die Geschwindigkeitsmessung, beziehen sich in der Regel auf die Messung von (sehr kurzen) Zeitintervallen, in denen ein lineares Verhalten angenommen wird. Das \emph{Sinnkriterium} zeigt sich aber auch in Begriffen wie \emph{lineares Fernsehen} oder \emph{lineare Erzählungen}. Dies ist zwar keine mathematische Linearität im Sinne der Formel \(f(x+y) = f(x) +f(y)\), aber der Begriff findet in einer verwandten Bedeutung in der Alltagssprache Verwendung.\index{Linearität|)}\index{Fundamentale Idee!Zeitkriterium|)}\index{Fundamentale Idee!Sinnkriterium|)}

\hypertarget{gegenbeispiele}{%
\subsection{Gegenbeispiele}\label{gegenbeispiele}}

Zur Verständnisförderung sollen noch ein paar Gegenbeispiele für Fundamentale Ideen angebracht werden.

\begin{itemize}
\tightlist
\item
  Das bereits erwähnte \textbf{Distributivgesetz} an sich ist zwar elementar, aber ihm fehlt die Weite, womit es nicht das Horizontalkriterium erfüllt. Die \emph{Linearität} als dahinterliegende Idee ist dagegen weit genug (vgl. ähnliche Argumentation zum \textbf{Kommutativgesetz} und der dahinterliegenden Idee der \emph{Invarianz} bei \protect\hyperlink{ref-Schubert:2011}{Schubert \& Schwill, 2011, S. 63}).
\item
  Der \textbf{Umkehrfunktion} fehlt das Sinnkriterium, da dieser Begriff in der Lebenswelt außerhab der Mathematik kaum von Relevanz ist. Dahinter liegt vielmehr die Idee der \emph{Reversibilität} als »Umkehrbarkeit von Operationen mit Wiederherstellung des Ausgangszustandes« (\protect\hyperlink{ref-Schubert:2011}{Schubert \& Schwill, 2011, S. 63}).
\end{itemize}

\hypertarget{mathematik-strukturieren-nachbereitung}{%
\section{Zum Nachbereiten}\label{mathematik-strukturieren-nachbereitung}}

\begin{enumerate}
\def\labelenumi{\arabic{enumi}.}
\tightlist
\item
  Lesen Sie sich die Bildungsstandards für den ersten/mittleren Schulabschluss vollständig durch, notieren Sie Verständnisfragen und bringen Sie diese zum Seminar mit.
\item
  Wählen Sie eine Leitidee aus und nennen Sie innerhalb dieser Begriffe, Sachverhalte und Verfahren, die im Mathematikunterricht behandelt werden.
\item
  Wählen Sie einen Begriff, Sachverhalt oder Verfahren aus 2. aus und stellen Sie den Bezug zu fundamentalen Ideen her.
\end{enumerate}

\hypertarget{appendix-anhang}{%
\appendix}


\hypertarget{vollstuxe4ndiges-literaturverzeichnis}{%
\chapter{Vollständiges Literaturverzeichnis}\label{vollstuxe4ndiges-literaturverzeichnis}}

\hypertarget{refs}{}
\begin{CSLReferences}{1}{0}
\leavevmode\vadjust pre{\hypertarget{ref-Ableitinger2013}{}}%
Ableitinger, C., Kramer, J., \& Prediger, S. (Hrsg.). (2013). \emph{Zur doppelten {Diskontinuität} in der {Gymnasiallehrerbildung}: {Ansätze} zu {Verknüpfungen} der fachinhaltlichen {Ausbildung} mit schulischen {Vorerfahrungen} und {Erfordernissen}}. Springer Fachmedien Wiesbaden. \url{https://doi.org/10.1007/978-3-658-01360-8}

\leavevmode\vadjust pre{\hypertarget{ref-Beutelspacher2012}{}}%
Beutelspacher, A., Danckwerts, R., Nickel, G., Spies, S., \& Wickel, G. (2012). \emph{Mathematik {Neu} {Denken}}. Vieweg+Teubner Verlag. \url{https://doi.org/10.1007/978-3-8348-8250-9}

\leavevmode\vadjust pre{\hypertarget{ref-Bruckler:2018}{}}%
Brückler, F. M. (2018). \emph{Geschichte der {Mathematik} kompakt: {Das} {Wichtigste} aus {Analysis}, {Wahrscheinlichkeitstheorie}, angewandter {Mathematik}, {Topologie} und {Mengenlehre}}. Springer Spektrum. \url{https://doi.org/10.1007/978-3-662-55574-3}

\leavevmode\vadjust pre{\hypertarget{ref-Bruner:1976}{}}%
Bruner, J. S. (1976). Die {Bedeutung} der {Struktur} im {Lernprozeß}. In A. Holtmann (Hrsg.), \emph{Das sozialwissenschaftliche {Curriculum} in der {Schule}: {Neue} {Formen} und {Inhalte}} (S. 77--90). VS Verlag für Sozialwissenschaften. \url{https://doi.org/10.1007/978-3-322-85275-5}

\leavevmode\vadjust pre{\hypertarget{ref-Danckwerts:1988}{}}%
Danckwerts, R. (1988). Linearität als organisierendes Element zentraler Inhalte der Schulmathematik. \emph{Didaktik der Mathematik}, \emph{16}(2), 149--160.

\leavevmode\vadjust pre{\hypertarget{ref-Danckwerts2013}{}}%
Danckwerts, R. (2013). Angehende {Gymnasiallehrer}(innen) brauchen eine „{Schulmathematik} vom höheren {Standpunkt}``! In C. Ableitinger, J. Kramer, \& S. Prediger (Hrsg.), \emph{Zur doppelten {Diskontinuität} in der {Gymnasiallehrerbildung}} (S. 77--94). Springer Fachmedien Wiesbaden. \url{https://doi.org/10.1007/978-3-658-01360-8_5}

\leavevmode\vadjust pre{\hypertarget{ref-Etzold:2019}{}}%
Etzold, H. (2019a). \emph{Winkel-{Farm}} (Version 2) {[}App{]}. \url{https://apps.apple.com/de/app/winkel-farm/id1369585218}

\leavevmode\vadjust pre{\hypertarget{ref-Etzold:2019Praxis4}{}}%
Etzold, H. (2019b). \emph{Winkel-{Farm} -- {Leitfaden} für {Lehrerinnen} und {Lehrer}} (Version 2). Zenodo. \url{https://doi.org/10.5281/zenodo.4747700}

\leavevmode\vadjust pre{\hypertarget{ref-Etzold2021}{}}%
Etzold, H. (2021). \emph{Neue Zugänge zum Winkelbegriff} {[}Dissertation, Universität Potsdam{]}. \url{https://doi.org/10.25932/publishup-50418}

\leavevmode\vadjust pre{\hypertarget{ref-Freudenthal1973c}{}}%
Freudenthal, H. (1973a). \emph{Mathematics as an {Educational} {Task}}. Springer Netherlands. \url{https://doi.org/10.1007/978-94-010-2903-2}

\leavevmode\vadjust pre{\hypertarget{ref-Freudenthal1973a}{}}%
Freudenthal, H. (1973c). \emph{Mathematik als pädagogische {Aufgabe}} (Bd. 1). Klett.

\leavevmode\vadjust pre{\hypertarget{ref-Freudenthal:1973}{}}%
Freudenthal, H. (1973b). \emph{Mathematik als pädagogische {Aufgabe}} (Bd. 2). Klett.

\leavevmode\vadjust pre{\hypertarget{ref-Gellert1986}{}}%
Gellert, W., Küstner, H., Hellwich, M., Kästner, H., \& Reichardt, H. (Hrsg.). (1986). \emph{Kleine {Enzyklopädie} {Mathematik}} (13. Aufl.). VEB Bibliographisches Institut.

\leavevmode\vadjust pre{\hypertarget{ref-Greefrath2016}{}}%
Greefrath, G., Oldenburg, R., Siller, H.-S., Ulm, V., \& Weigand, H.-G. (2016). \emph{Didaktik der {Analysis}. {Aspekte} und {Grundvorstellungen} zentraler {Begriffe}} (F. Padberg \& A. Büchter, Hrsg.; 4. Aufl.). Springer Spektrum. \url{https://doi.org/10.1007/978-3-662-48877-5}

\leavevmode\vadjust pre{\hypertarget{ref-Hefendehl-Hebeker:2016}{}}%
Hefendehl-Hebeker, L. (2016). Subject-matter didactics in {German} traditions: {Early} historical developments. \emph{Journal für Mathematik-Didaktik}, \emph{37}(S1), 11--31. \url{https://doi.org/10.1007/s13138-016-0103-7}

\leavevmode\vadjust pre{\hypertarget{ref-Henn2015}{}}%
Henn, H.-W., \& Filler, A. (2015). \emph{Didaktik der {Analytischen} {Geometrie} und {Linearen} {Algebra}: {Algebraisch} verstehen -- {Geometrisch} veranschaulichen und anwenden}. Springer Spektrum.

\leavevmode\vadjust pre{\hypertarget{ref-Hussmann:2016}{}}%
Hußmann, S., \& Prediger, S. (2016). Specifying and Structuring Mathematical Topics: A Four-Level Approach for Combining Formal, Semantic, Concrete, and Empirical Levels Exemplified for Exponential Growth. \emph{Journal für Mathematik-Didaktik}, \emph{37}(S1), 33--67. \url{https://doi.org/10.1007/s13138-016-0102-8}

\leavevmode\vadjust pre{\hypertarget{ref-Hussmann:2016a}{}}%
Hußmann, S., Rezat, S., \& Sträßer, R. (2016). Subject {Matter} {Didactics} in {Mathematics} {Education}. \emph{Journal für Mathematik-Didaktik}, \emph{37}(S1), 1--9. \url{https://doi.org/10.1007/s13138-016-0105-5}

\leavevmode\vadjust pre{\hypertarget{ref-Jahnke:2010}{}}%
Jahnke, T. (2010). Vom mählichen {Verschwinden} des {Fachs} aus der {Mathematikdidaktik}. \emph{GDM-Mitteilungen 89}, 21--24. \url{https://ojs.didaktik-der-mathematik.de/index.php/mgdm/article/view/559/550}

\leavevmode\vadjust pre{\hypertarget{ref-Klein1925}{}}%
Klein, F. (1925). \emph{Elementarmathematik vom {Höheren} {Standpunkte} aus {II}. {Geometrie}}. Springer Berlin Heidelberg. \url{https://doi.org/10.1007/978-3-642-90852-1}

\leavevmode\vadjust pre{\hypertarget{ref-Klein1955}{}}%
Klein, F. (1955). \emph{Elementarmathematik vom {Höheren} {Standpunkte} aus {III}. {Präzisions}- und {Approximationsmathematik}} (C. H. Müller, Hrsg.). Springer Berlin Heidelberg. \url{https://doi.org/10.1007/978-3-662-00246-9}

\leavevmode\vadjust pre{\hypertarget{ref-Klein1967}{}}%
Klein, F. (1967). \emph{Elementarmathematik vom {Höheren} {Standpunkte} aus {I}. {Arithmetik}, {Algebra}, {Analysis}}. Springer Berlin Heidelberg. \url{https://doi.org/10.1007/978-3-662-11652-4}

\leavevmode\vadjust pre{\hypertarget{ref-Krainer:1989}{}}%
Krainer, K. (1989). \emph{Lebendige {Geometrie}. Überlegungen zu einem integrativen {Verständnis} von {Geometrieunterricht} anhand des {Winkelbegriffs}} {[}Dissertation{]}. Alpen-Adria-Universität Klagenfurt.

\leavevmode\vadjust pre{\hypertarget{ref-Kruger2015}{}}%
Krüger, K., Sill, H.-D., \& Sikora, C. (2015). \emph{Didaktik der {Stochastik} in der {Sekundarstufe} {I}}. Springer Berlin Heidelberg. \url{https://doi.org/10.1007/978-3-662-43355-3}

\leavevmode\vadjust pre{\hypertarget{ref-Lambert:2012}{}}%
Lambert, A. (2012). \emph{Gedanken zum aktuellen {Kompetenzbegriff} für den ({Mathematik}-)unterricht} {[}Vortrag{]}. Eingangsstatement zur Podiumsdiskussion im Rahmen des 3. Fachdidaktischen Kolloquiums an der Universität des Saarlandes, Saarbrücken. \url{https://www.uni-saarland.de/fileadmin/upload/einrichtung/zfl/PDF_Fachdidaktik/PDF_Kolloquium_FD/Kompetenzbegriff_für_den_Mathematikunterricht_Statement_mit_Folien.pdf}

\leavevmode\vadjust pre{\hypertarget{ref-Lompscher1985b}{}}%
Lompscher, J. (1985). Die {Lerntätigkeit} als dominierende {Tätigkeit} des jüngeren {Schülers}. In J. Lompscher (Hrsg.), \emph{Persönlichkeitsentwicklung in der {Lerntätigkeit}} (S. 23--52). Volk und Wissen.

\leavevmode\vadjust pre{\hypertarget{ref-MinisteriumfurBildungJugendundSportdesLandesBrandenburg2022}{}}%
Ministerium für Bildung, J. und S. des L. B. (Hrsg.). (2022). \emph{Rahmenlehrplan für die gymnasiale {Oberstufe}. {Teil} {C}. {Mathematik}}. \url{https://bildungsserver.berlin-brandenburg.de/fileadmin/bbb/unterricht/rahmenlehrplaene/gymnasiale_oberstufe/curricula/2022/Teil_C_RLP_GOST_2022_Mathematik.pdf}

\leavevmode\vadjust pre{\hypertarget{ref-MinisteriumfuerBildungJugendundSportdesLandesBrandenburg2023}{}}%
Ministerium für Bildung, J. und S. des L. B. (Hrsg.). (2023). \emph{Rahmenlehrplan {Brandenburg}. {Teil} {C}, {Mathematik}, {Jahrgangsstufen} 1~--~10}. \url{https://bildungsserver.berlin-brandenburg.de/fileadmin/bbb/unterricht/rahmenlehrplaene/Rahmenlehrplanprojekt/amtliche_Fassung/getrennt_2023/BB_RLP_2023_Teil_C_Ma_GenF_1.pdf}

\leavevmode\vadjust pre{\hypertarget{ref-Mitchelmore:1990}{}}%
Mitchelmore, M. (1990). Psychologische und mathematische Schwierigkeiten beim Lernen des Winkelbegriffs. \emph{mathematica didactica}, \emph{13}, 19--37.

\leavevmode\vadjust pre{\hypertarget{ref-Mitchelmore:1998}{}}%
Mitchelmore, M., \& White, P. (1998). Development of {Angle} {Concepts}: {A} {Framework} for {Research}. \emph{Mathematics Education Research Journal}, \emph{10}(3), 4--27.

\leavevmode\vadjust pre{\hypertarget{ref-Schubert:2011}{}}%
Schubert, S., \& Schwill, A. (2011). \emph{Didaktik der {Informatik}} (2. Aufl). Spektrum, Akad. Verl. \url{https://doi.org/10.1007/978-3-8274-2653-6}

\leavevmode\vadjust pre{\hypertarget{ref-Schupp:2016}{}}%
Schupp, H. (2016). Gedanken zum „{Stoff}`` und zur „{Stoffdidaktik}`` sowie zu ihrer {Bedeutung} für die {Qualität} des {Mathematikunterrichts}. \emph{Mathematische Semesterberichte}, \emph{63}(1), 69--92. \url{https://doi.org/10.1007/s00591-016-0159-y}

\leavevmode\vadjust pre{\hypertarget{ref-Schwill:1994}{}}%
Schwill, A. (1994). \emph{Fundamentale {Ideen} in {Mathematik} und {Informatik}}. Herbsttagung des Arbeitskreises Mathematikunterricht und Informatik, Wolfenbüttel. \url{http://www.informatikdidaktik.de/didaktik/Forschung/Wolfenbuettel94.pdf}

\leavevmode\vadjust pre{\hypertarget{ref-KMK:2012}{}}%
Sekretariat der Ständigen Konferenz der Kultusminister der Länder in der Bundesrepublik Deutschland. (2012). \emph{Bildungsstandards im {Fach} {Mathematik} für die {Allgemeine} {Hochschulreife}. (Beschluss der Kultusministerkonferenz vom 18.10.2012)}. \url{https://www.kmk.org/fileadmin/Dateien/veroeffentlichungen_beschluesse/2012/2012_10_18-Bildungsstandards-Mathe-Abi.pdf}

\leavevmode\vadjust pre{\hypertarget{ref-SekretariatderStandigenKonferenzderKultusministerderLanderinderBundesrepublikDeutschland2022}{}}%
Sekretariat der Ständigen Konferenz der Kultusminister der Länder in der Bundesrepublik Deutschland. (2022a). \emph{Bildungsstandards für das {Fach} {Mathematik} {Erster} {Schulabschluss} ({ESA}) und {Mittlerer} {Schulabschluss} ({MSA}). ({Beschluss} der {Kultusministerkonferenz} vom 15.10.2004 und vom 04.12.2003, i.d.{F}. vom 23.06.2022)}. \url{https://www.kmk.org/fileadmin/Dateien/veroeffentlichungen_beschluesse/2022/2022_06_23-Bista-ESA-MSA-Mathe.pdf}

\leavevmode\vadjust pre{\hypertarget{ref-SekretariatderStandigenKonferenzderKultusministerderLanderinderBundesrepublikDeutschland2022a}{}}%
Sekretariat der Ständigen Konferenz der Kultusminister der Länder in der Bundesrepublik Deutschland. (2022b). \emph{Bildungsstandards für das {Fach} {Mathematik} {Primarbereich}. ({Beschluss} der {Kultusministerkonferenz} vom 15.10.2004, i.d.{F}. vom 23.06.2022)}. \url{https://www.kmk.org/fileadmin/Dateien/veroeffentlichungen_beschluesse/2022/2022_06_23-Bista-Primarbereich-Mathe.pdf}

\leavevmode\vadjust pre{\hypertarget{ref-Steinhofel1988}{}}%
Steinhöfel, W., Reichold, K., \& Frenzel, L. (1988). \emph{Zur {Gestaltung} typischer {Unterrichtssituationen} im {Mathematikunterricht}}. Ministerium für Volksbildung.

\leavevmode\vadjust pre{\hypertarget{ref-Strehl:1983}{}}%
Strehl, R. (1983). Anschauliche {Vorstellung} und mathematische {Theorie} beim {Winkelbegriff}. \emph{mathematica didactica}, \emph{6}, 129--146.

\leavevmode\vadjust pre{\hypertarget{ref-Taylor:1715}{}}%
Taylor, B. (1715). \emph{Linear perspective}. printed for R. Knaplock at the Bishop's-Head in St. Paul's Church-Yard. \url{https://nl.sub.uni-goettingen.de/id/0590700700}

\leavevmode\vadjust pre{\hypertarget{ref-Tietze:2000a}{}}%
Tietze, U.-P., Klika, M., \& Wolpers, H. (Hrsg.). (2000a). \emph{Mathematikunterricht in der {Sekundarstufe} {II}. {Band} 1: {Fachdidaktische} {Grundfragen}, {Didaktik} der {Analysis}} (2. Aufl.). Vieweg+Teubner Verlag. \url{https://doi.org/10.1007/978-3-322-90568-0}

\leavevmode\vadjust pre{\hypertarget{ref-Tietze:2000}{}}%
Tietze, U.-P., Klika, M., \& Wolpers, H. (Hrsg.). (2000b). \emph{Mathematikunterricht in der {Sekundarstufe} {II}. {Band} 2: {Didaktik} der {Analytischen} {Geometrie} und {Linearen} {Algebra}}. Vieweg+Teubner Verlag. \url{https://doi.org/10.1007/978-3-322-86479-6}

\leavevmode\vadjust pre{\hypertarget{ref-Tietze:2002}{}}%
Tietze, U.-P., Klika, M., \& Wolpers, H. (Hrsg.). (2002). \emph{Mathematikunterricht in der {Sekundarstufe} {II}. {Band} 3: {Didaktik} der {Stochastik}}. Vieweg+Teubner Verlag. \url{https://doi.org/10.1007/978-3-322-83144-6}

\leavevmode\vadjust pre{\hypertarget{ref-Vohns:2000}{}}%
Vohns, A. (2000). \emph{Das {Messen} als fundamentale {Idee}} {[}1. Staatsexamensarbeit, Universität-Gesamthochschule Siegen{]}. \url{https://wwwu.aau.at/avohns/pdf/messen.pdf}

\leavevmode\vadjust pre{\hypertarget{ref-Vollrath2012}{}}%
Vollrath, H.-J., \& Roth, J. (2012). \emph{Grundlagen des {Mathematikunterrichts} in der {Sekundarstufe}} (F. Padberg, Hrsg.; 2. Aufl.). Spektrum Akademischer Verlag. \url{https://doi.org/10.1007/978-3-8274-2855-4}

\leavevmode\vadjust pre{\hypertarget{ref-vonderBank:2013}{}}%
von der Bank, M.-C. (2013). Fundamentale {Ideen}, insbesondere {Optimierung}. In A. Filler \& M. Ludwig (Hrsg.), \emph{Wege zur {Begriffsbildung} für den {Geometrieunterricht}. {Ziele} und {Visionen} 2020. {Vorträge} auf der 29. {Herbsttagung} des {Arbeitskreises} {Geometrie} in der {Gesellschaft} für {Didaktik} der {Mathematik} vom 14. bis 16. {September} 2012 in {Saarbrücken}} (S. 83--124). Franzbecker. \url{https://www.math.uni-sb.de/service/lehramt/AKGeometrie/AKGeometrie2012.pdf}

\leavevmode\vadjust pre{\hypertarget{ref-Bank:2016}{}}%
von der Bank, M.-C. (2016). \emph{Fundamentale {Ideen} der {Mathematik}: {Weiterentwicklung} einer {Theorie} zu deren unterrichtspraktischer {Nutzung}} {[}Dissertation, Universität des Saarlandes{]}. \url{https://doi.org/10.22028/D291-26673}

\leavevmode\vadjust pre{\hypertarget{ref-Weigand2018}{}}%
Weigand, H.-G., Filler, A., Hölzl, R., Kuntze, S., Ludwig, M., Roth, J., Schmidt-Thieme, B., \& Wittmann, G. (2018). \emph{Didaktik der {Geometrie} für die {Sekundarstufe} {I}}. Springer Berlin Heidelberg. \url{https://doi.org/10.1007/978-3-662-56217-8}

\leavevmode\vadjust pre{\hypertarget{ref-Weigand2022}{}}%
Weigand, H.-G., Schüler-Meyer, A., \& Pinkernell, G. (2022). \emph{Didaktik der {Algebra}: nach der {Vorlage} von {Hans}-{Joachim} {Vollrath}}. Springer Berlin Heidelberg. \url{https://doi.org/10.1007/978-3-662-64660-1}

\leavevmode\vadjust pre{\hypertarget{ref-WikiPeano}{}}%
Wikipedia. (2021). \emph{Peano-Axiome --- Wikipedia{,} die freie Enzyklopädie}. \url{https://de.wikipedia.org/w/index.php?title=Peano-Axiome\&oldid=216675163}

\leavevmode\vadjust pre{\hypertarget{ref-dewiki:228417777}{}}%
Wikipedia. (2022). \emph{Kultusministerkonferenz --- Wikipedia{,} die freie Enzyklopädie}. \url{https://de.wikipedia.org/w/index.php?title=Kultusministerkonferenz\&oldid=228417777}

\leavevmode\vadjust pre{\hypertarget{ref-Wittmann:2015}{}}%
Wittmann, E. C. (2015). Strukturgenetische didaktische {Analysen} -- empirische {Forschung} „erster {Art}``. \emph{mathematica didactica}, 239--255. \url{http://www.mathematica-didactica.com/altejahrgaenge/md_2015/md_2015_Wittmann_Stoffdidaktik.pdf}

\end{CSLReferences}

% \printindex

\end{document}
